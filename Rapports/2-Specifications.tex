\documentclass[a4paper,11pt]{article}
\usepackage{exptech}
\usepackage{textcomp}
\usepackage{graphicx}
\usepackage{array}
\usepackage[babel=true]{csquotes}
\usepackage{url}
\usepackage{hyperref}
\usepackage{wrapfig}
\usepackage[export]{adjustbox}
\usepackage{titletoc}

\titlecontents{subsection}[3.8em]{}{}{}{}[\addvspace{-0.5pt}]
\hypersetup{
  bookmarks=true, % show bookmarks bar?
  pdftitle={Avalon - Rapport de spécifications fonctionnelles}, % title
  pdfnewwindow=true, % links in new window
  colorlinks=true, % false: boxed links; true: colored links
  linkcolor=black, % color of internal links (change box color with linkbordercolor)
  citecolor=cyan, % color of links to bibliography
  filecolor=cyan, % color of file links
  urlcolor=cyan % color of external links
}

\title{
  \textbf{Avalon}\\
  Rapport de spécifications fonctionnelles
}
\markright{Avalon - Rapport de spécifications fonctionnelles}
\author{
\begin{minipage}{0.4\textwidth}
	\begin{flushleft} \large
		\emph{Auteurs :}\\
		Alexandre \textsc{Audinot}\\
		Julien \textsc{Bouvet}\\
		Cyrille \textsc{Delabre}\\
		Thierry \textsc{Gaugry}\\
		Nicolas \textsc{Hurman}\\
		Léo \textsc{Jacoboni}\\
		Alexandre \textsc{Leonardi}\\
	\end{flushleft}
\end{minipage}
\begin{minipage}{0.4\textwidth}
	\begin{flushright} \large
		\emph{Encadrants :} \\
		Valérie \textsc{Gouranton}\\
		Ronan \textsc{Gaugne}\\
		Bruno \textsc{Arnaldi}\\
		Willy \textsc{Allègre}\\
		Jean-Paul  \textsc{Departe}\\
	\end{flushright}
\end{minipage}
}

\date{23 Octobre 2014}

\begin{document}
\maketitle
\thispagestyle{empty}
\begin{abstract}
\textbf{Avalon :} Environnement de Réalité Virtuelle pour l'apprentissage à l'utilisation d'appartements tremplins. Réalisation en 3D d'un appartement domotisé interactif utilisé dans le cadre de la rééducation des personnes handicapées.
Le projet est proposé par le centre mutualiste de rééducation et de réadaptation fonctionnelles de Kerpape (plus particulièrement les ingénieurs du laboratoire électronique Willy Allègre et Jean-Paul Departe).
Le modèle 3D de l'appartement nous est fourni, et notre travail consistera à réaliser un logiciel fonctionnel permettant de se déplacer dans l'appartement et implémentant les interactions avec les différents éléments de domotiques, en plus de prendre en charge différents périphériques de contrôle. 
\end{abstract}

\begin{figure}[h!]
	\centering
	\includegraphics[height=170pt]{2-Specifications/img/screen_appart.png}
\end{figure}

%\vfill
%[width=\textwidth]
\begin{figure}[h!]
   \begin{minipage}{0.3\linewidth}
      \includegraphics[scale=0.9]{2-Specifications/img/logo_insa.jpeg}
   \end{minipage} 
   \begin{minipage}{0.2\linewidth}
      \centering
      \includegraphics[scale=0.5,left]{2-Specifications/img/logo_irisa.jpg}
   \end{minipage}\hfill
   \begin{minipage}{0.2\linewidth}
      \includegraphics[scale=0.9]{2-Specifications/img/logo_kerpape.png}
   \end{minipage}
\end{figure}

\pagebreak

\tableofcontents
\pagebreak


\section{Introduction}



Notre projet consiste en la simulation d'appartements tremplins (fig. \ref{appart}) en 3D pleinement interactifs, dans lesquels des personnes handicapées pourront évoluer et apprendre à se servir des différents équipements de domotique présents. L'objectif de ces appartements est de permettre à ces personnes de devenir autonomes en les accompagnant dans cet apprentissage.
Le projet est proposé par le centre de rééducation de Kerpape et par l'IRISA, ce qui lui confère un intérêt particulier à nos yeux car le commanditaire est extérieur à l'école, et le projet répond donc au besoin d'un véritable client de la même manière que le ferait un projet rencontré dans notre vie professionnelle. 
Ainsi, nous avons eu l'occasion de nous entretenir avec Willy Allègre et Jean-Paul Departe, ingénieurs de Kerpape à l'origine du projet, tout d'abord lors d'une conférence téléphonique pendant laquelle ils nous ont décrit ce qu'ils attendaient de l'application. Par la suite, un déplacement au centre de Kerpape nous a permis de parfaire l'image que nous nous faisons du résultat attendu et de compléter le cahier des charges de la future application. 
De plus ce projet présente un véritable intérêt social, puisque l'application finale permettra aux patients de redevenir autonomes et d'apprendre à vivre avec leur handicap.


\begin{figure}[h]
	\centering
	\includegraphics[scale=1]{1-PreEtude/img/appt_tremplin_intro.png}
	\caption{Appartement tremplin : chambre}
	\label{appart}
\end{figure}

Lors de ce rapport, nous allons étudier les spécifications de notre projet, nous allons pour cela les séparer en deux catégories. Tout d'abord, les spécifications fonctionnelles du projet, c'est-à-dire le contenu du logiciel tels que la gestion de périphériques spécifiques ou les différents scénarios. Ensuite, celles au niveau de l'utilisateur, il s'agit des contraintes quant à l'ergonomie pour les patients ainsi que pour le thérapeute. Nous allons enfin faire une ébauche de l'architecture générale du logiciel, où nous ferons une liste des différentes briques logicielles que nous utiliserons et détaillerons comment les elles sont reliées.



\pagebreak
\section{Spécifications fonctionnelles du projet}

\pagebreak
\section{Spécifications niveau utilisateur}

\pagebreak
\section{Ebauche de l'architecture logicielle générale}

\pagebreak
\section{Conclusion}
Notre logiciel est destiné à un largé éventail de personnes ; n'ayant pas forcément des compétences en informatique très développées.
\`A travers les diagrammes de cas d'utilisation et d'interaction, nous avons détaillé une ergonomie qui nous semble suffisament accessible pour satisfaire les utilisateurs.

Concernant le fonctionnement du logiciel, il sera donc basé sur Unity auquel nous ajouterons des scripts C\#.

Maintenant que nous avons éclairci les objectifs du logiciel, nous nous baserons sur ce rapport comme cahier des charges de référence. Il est évident que n'ayant pas encore pris en main l'ensemble des outils de développement, certains points seront probablement modifiés par la suite.



\subsection{Versions intermédiaires}
Lors de la réalisation de ce projet, nous allons produire plusieurs versions intermédiaires pour nous permettre de construire chaque fonctionnalité au fur et à mesure :
\begin{itemize}
  \item Version \textnumero1 - 9 décembre : Implémentation de l'appartement complet dans un environement 3d. Le personnage peut se déplacer librement à l'intérieur, mais sans collisions avec l'environnement.
  \item Version \textnumero2 - 9 janvier : Une deuxième version ajoute les interactions basiques : collision avec les obstacles, possibilité d'utiliser les interrupteurs. Elle correspond au mode \enquote{Utilisation} du logiciel.
  \item Version \textnumero3 - 9 février :  La troisième version permet d'utiliser le mode \enquote{Apprentissage}.
  \item Version \textnumero4 - 9 mars : La quatrième version implémente les différents scénarios d'utilisation du logiciel et fonctionne avec PC+clavier/souris.
  \item Version \textnumero5 - 9 avril : La version finale fonctionne en réalité virtuelle dans ces 3 envionnements : salle $\mu$RV, salle Immersia, casque de réalité virtuelle.
\end{itemize}

\pagebreak
\bibliography{2-Specifications/biblio}

\end{document}
