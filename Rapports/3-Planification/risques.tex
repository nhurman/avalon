\section{Évaluation des risques}

Il s'agit de la dernière étape de la planification du projet Avalon : l’évaluation des risques qui pourraient survenir, ainsi que la solution à adopter le cas échéant. Nous les avons classés en 3 catégories, les risques humains, les risques techniques, et ceux liés à une mauvaise gestion du projet. 

\subsection{Risques humains}
Des erreurs ou des actions malintentionnées que quelqu'un pourrait commettre. 

\textbf{Vol de matériel}

C'est sans doute le risque le plus grave mais aussi le moins probable. Un vol de matériel, par exemple le vol de notre Oculus Rift, serait extrêmement préjudiciable : le matériel est cher et compliqué à remplacer, mais aussi nécessaire pour la réalisation du projet et la conduite de tests. Dans ce cas-là, nous demanderions au département Informatique ou à l'IRISA de nous prêter le matériel manquant.

Néanmoins, ce cas de figure a peu de chances de se réaliser car la salle $\mu$RV est verrouillée dès que personne ne l'occupe.

\textbf{Perte de données}

Éventualité déjà plus probable, la perte de données est censée être peut s'avérer elle aussi assez handicapante, en fonction de la quantité de données ainsi disparues. Cependant, l'utilisation d'un gestionnaire de versions est censée prévenir ce type de déboires.

Il peut toujours arriver que cela ne suffise pas, et dans ce cas-là il n'y a guère de solutions pour récupérer ce qui a été perdu. Cependant, le \emph{commit} régulier des modifications fait que ce ne sont que de petites portions de codes qui sont sujettes à ce risque.

\subsection{Risques techniques}
Ce sont les risques ne dépendant pas des membres du groupe ou d'une autre personne. Ils sont liés aux machines ou logiciels utilisés.

\textbf{Dépôt Git indisponible}

Nous utilisons Git comme gestionnaire de versions pour le travail collaboratif. Cependant, chaque membre possédant une copie du projet au complet sur son disque dur, nous pourrions changer de gestionnaire de versions dans le cas d'un problème persistant ou partager le code directement \emph{via} clé USB dans le cas d'un problème ponctuel.

\textbf{Pannes}

Les pannes de matériel peuvent avoir une incidence variable. Une panne d'un PC sera sans impact car nous en avons suffisamment à disposition pour que cela ne nous retarde pas ; la panne d'un dispositif unique comme l'Oculus Rift s'avérera en revanche plus problématique.

Dans ce cas comme dans le cas du vol, la solution consiste à demander au département informatique ou à l'IRISA de nous prêter du matériel pour remplacer celui qui est en panne.

\subsection{Risques quant à la gestion du projet}
Une mauvaise gestion du projet pourrait conduire à des problèmes épineux. Il peut être compliqué de découvrir le problème avant qu'il ait pris une certaine ampleur. 

\textbf{Clients insatisfaits}

Dans le cas où l'application finale ne conviendrait pas au centre de Kerpape, il pourrait être trop tard pour effectuer les changements qui seraient attendus. C'est la raison pour laquelle nous avons décidé de présenter plusieurs versions à Jean-Paul Departe et Willy Allègre. Ainsi, si une partie du projet ne leur convient pas, nous sommes sûrs de le détecter suffisamment tôt. 

\textbf{Sous-estimation de la complexité d'une tâche}

Un risque inhérent à notre manque d'expérience avec des projets de cette ampleur : il est parfois difficile d'estimer la durée que prendra une tâche et les difficultés qu'elle présentera. Notre planning tient compte de cette possibilité et les estimations de temps ont été faites en conséquence.
