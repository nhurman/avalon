\section{Méthodes d'organisation \& planification}

Au vu de l'envergure de notre projet, nous sommes obligés de prévoir une planification la plus détaillée possible, qui tienne compte de toutes les tâches qui compoeront la réalisation de notre application, accompagnées de la charge de travail que nous estimons devoir fournir pour les réaliser. Les différents livrables qui nous sont demandés feront office de jalons tout au long de notre projet. 

\subsection{Planification prévisionnelle}

Voici le planing, tel que nous l'avions estimé lors de notre 1\up{er} rapport : \newline
\begin{tabular}{|l|l|}
\hline
  Date &
  Production \\
\hline
  9 janvier &
  Version PC \textnumero2 \\
\hline
  6 février &
  Rapport de conception logicielle VP \\
\hline
  9 février &
  Version PC \textnumero3 \\
\hline
  12 février &
  Rapport de conception logicielle VF \\
\hline
  9 mars &
  Version PC \textnumero4 \\
\hline
  25 mars &
  Page HTML VP \\
\hline
  2 avril &
  Page HTML VF \\
\hline
  9 avril &
  Version PC \textnumero5 \\
\hline
  16 mai &
  Documentation en Ligne VP \\
\hline
  21 mai &
  Rapport Final/Annexes + Bilan Planification \\
\hline
  26 mai &
  Rapport Final/Annexes + Bilan Planification \\
\hline
  28 mai &
  Documentation en Ligne VF \\
\hline
\end{tabular}

\subsection{Répartition des temps de travail}

\subsection{Méthode agile \& méthode en V}
