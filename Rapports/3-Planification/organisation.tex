\section{Méthodes d'organisation \& planification}
Au vu de l'envergure de notre projet, nous sommes obligés de prévoir une planification la plus détaillée possible, qui tienne compte de toutes les tâches qui compoeront la réalisation de notre application, accompagnées de la charge de travail que nous estimons devoir fournir pour les réaliser. Les différents livrables qui nous sont demandés feront office de jalons tout au long de notre projet. 

\subsection{Planification prévisionnelle}
Voici le planing, tel que nous l'avions estimé lors de notre 1\up{er} rapport : \newline
\begin{tabular}{|l|l|}
\hline
  Date &
  Production \\
\hline
  9 janvier &
  Version PC \textnumero2 \\
\hline
  6 février &
  Rapport de conception logicielle VP \\
\hline
  9 février &
  Version PC \textnumero3 \\
\hline
  12 février &
  Rapport de conception logicielle VF \\
\hline
  9 mars &
  Version PC \textnumero4 \\
\hline
  25 mars &
  Page HTML VP \\
\hline
  2 avril &
  Page HTML VF \\
\hline
  9 avril &
  Version PC \textnumero5 \\
\hline
  16 mai &
  Documentation en Ligne VP \\
\hline
  21 mai &
  Rapport Final/Annexes + Bilan Planification \\
\hline
  26 mai &
  Rapport Final/Annexes + Bilan Planification \\
\hline
  28 mai &
  Documentation en Ligne VF \\
\hline
\end{tabular}

\subsection{Méthode agile \& méthode en V}
Au cours du projet, 

Pour ce projet, nous devons produire plusieurs livrables : six rapports, une
documentation en ligne, une page HTML résumant notre projet ainsi qu’un ou plusieurs
exécutables, leur procédure d’installation, et enfin des jeux de tests.
Ces livrables sont dus à des dates dites « jalons », après les phases
correspondantes. Par exemple, le rapport de conception sera à rendre le 13 février,
après une phase de conception durant du 19 décembre au 13 février.
Ces dates « jalons » ainsi que les différentes phases sont définies de façon
commune pour tous les projets : nous ne pouvons pas passer outre et devons respecter
les délais imposés. C’est pourquoi nous avons préféré, après d’âpres discussions,
adopter une méthode de production en V (cf. Figure 3) plutôt qu’une méthode agile
avec des « sprints » qui nous semblait peu réalisable au vu de la durée probable de
ces sprints (un mois). De plus, des sprints d’un mois impliqueraient de faire de la
conception de façon permanente, alors que nous avons une phase de conception
prévue dans la définition du projet pour cela.

\subsection{Répartition des temps de travail}


