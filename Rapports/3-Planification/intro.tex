\section{Introduction}
Dans le cadre de notre projet 4INFO, nous allons réaliser une application pour le Centre de de rééducation fonctionnelle de Kerpape. Il s'agit de modéliser des appartements tremplins qui sont spécialement équipés pour que les personnes handicapées puissent y vivre en toute autonomie. Ils servent à les préparer à reprendre une vie normale dans des appartements lourdement équipés en domotique. Le problème est que le centre ne dispose que de deux appartements tremplins, ce qui n'est pas suffisant pour le nombre de leurs patients.\newline

De plus, le nombre important des équipements automatisés pose des problèmes d'apprentissage, particulièrement pour les personnes ayant des troubles cognitifs. De ce fait, l'objectif de notre application est de fournir aux patients, ainsi qu'aux thérapeutes qui les accompagnent, un outil informatique qui leur permettra de se familiariser avec les équipements des appartements avant le test \textit{in situ}. \newline

Notre application implémentera notamment les fonctionnalités suivantes :
\begin{itemize}
	\item 3 modes d'apprentissage, chacun avec un niveau de réalisme différent, qui permettent d'aller d'une prise en main basique à une totale liberté
	\item 3 scénarios prédéfinis mettant en scène l'utilisation du combiné téléphone/interphone
 	\item Le support de différents périphériques d'utilisation (clavier/souris, Oculus Rift, salle de réalité virtuelle Immersia)
\end{itemize}
\vspace{0.5cm}
\hspace{0.5cm}L'ensemble de nos spécifications (détaillées dans le rapport précédent) nous a permis d'aiguiller la planification de ce projet présentée dans ce rapport.