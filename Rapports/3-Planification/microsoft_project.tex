\section{Planification Microsoft Project}

Microsoft Project est un outil puissant, que nous apprenons tout juste à appréhender. Il nous offre un grand nombre de possibilités ainsi qu'une aide qui s'avérera utile dans le pilotage de notre projet. Nous avons retenu les informations les plus importantes qu'il synthétise.

\subsection{Chronologie du projet}
Cette vue est la plus synthétique que nous ayons réalisée, elle donne une vue d’ensemble du projet ne comprenant que les plus grandes étapes. \newline
\emph{Mettre ici chronologie générale}\newline

Une autre vue importante tout en restant synthétique est la liste des jalons qui structurent le développement de notre application.\newline
\emph{Mettre les jalons à respecter}

\subsection{Affectation des ressources par tâche}
Nous avons subdivisé notre projet en un ensemble de sous-tâches avec le plus de précision possible, en tentant de faire une estimation de la durée que chacune d'entre elle demanderait.\newline

Le diagramme suivant résume le temps requis par chaque tâche, en prenant en compte notre nombre ainsi que le travail que nous pourrons fournir selon les semaines (semaine d'examens, semaine réservée au projet, etc).\newline
\emph{Mettre la répartition du temps de travail par partie}

\subsection{Diagramme de Gantt}
Le diagramme de Gantt permet d'avoir une vue plus détaillée des différentes tâches du projet, ordonnancées en fonction de leur durée et de leur priorité. \newline
\emph{Mettre le diagramme de Gantt}