\documentclass[a4paper,11pt]{article}

\usepackage{exptech}
\usepackage{textcomp}
\usepackage{graphicx}
\usepackage{array}
\usepackage[babel=true]{csquotes}
\usepackage{url}
\usepackage{hyperref}
\hypersetup{
  bookmarks=true, % show bookmarks bar?
  pdftitle={Avalon - Rapport de pré-étude}, % title
  pdfnewwindow=true, % links in new window
  colorlinks=true, % false: boxed links; true: colored links
  linkcolor=black, % color of internal links (change box color with linkbordercolor)
  citecolor=cyan, % color of links to bibliography
  filecolor=cyan, % color of file links
  urlcolor=cyan % color of external links
}

\title{
  \textbf{Avalon}\\
  Rapport de pré-étude
}
\markright{Avalon - Rapport de pré-étude}
\author{
\begin{minipage}{0.4\textwidth}
	\begin{flushleft} \large
		\emph{Auteurs :}\\
		Alexandre \textsc{Audinot}\\
		Julien \textsc{Bouvet}\\
		Cyrille \textsc{Delabre}\\
		Thierry \textsc{Gaugry}\\
		Nicolas \textsc{Hurman}\\
		Léo \textsc{Jacoboni}\\
		Alexandre \textsc{Leonardi}\\
	\end{flushleft}
\end{minipage}
\begin{minipage}{0.4\textwidth}
	\begin{flushright} \large
		\emph{Encadrants :} \\
		Valérie \textsc{Gouranton}\\
		Ronan \textsc{Gaugne}\\
		Bruno \textsc{Arnaldi}\\
		Willy \textsc{Allègre}\\
		Jean-Paul  \textsc{Departe}\\
	\end{flushright}
\end{minipage}
}

\date{23 Octobre 2014}

\begin{document}
\maketitle
\thispagestyle{empty}
\begin{abstract}
\textbf{Avalon :} Environnement de Réalité Virtuelle pour l'apprentissage à l'utilisation d'appartements tremplins. Réalisation en 3D d'un appartement domotisé interactif utilisé dans le cadre de la rééducation des personnes handicapées.
Le projet est proposé par le centre mutualiste de rééducation et de réadaptation fonctionnelles de Kerpape (plus particulièrement Willy Allègre et Jean-Paul Departe) et sera réalisé à l'aide de Unity3D et MiddleVR.
Nous devons intégrer le modèle dans un moteur 3D et réaliser plusieurs scénarios mettant en valeur les éléments que nous aurons rendus interactifs, comme les divers panneaux de commande, pour permettre aux utilisateurs de se familiariser avec le fonctionnement de la domotique des lieux.
\end{abstract}

\begin{figure}[h!]
	\centering
	\includegraphics[height=150pt]{1-PreEtude/img/screen_appart.png}
\end{figure}

\vfill

\begin{figure}[h!]
   \begin{minipage}{0.2\linewidth}
      \includegraphics[width=\textwidth]{1-PreEtude/img/logo_insa.jpeg}
   \end{minipage} \hfill
   \begin{minipage}{0.2\linewidth}
      \includegraphics[width=\textwidth]{1-PreEtude/img/logo_irisa.jpg}
   \end{minipage} \hfill
   \begin{minipage}{0.2\linewidth}
      \includegraphics[width=\textwidth]{1-PreEtude/img/logo_kerpape.png}
   \end{minipage}
\end{figure}

\pagebreak

\tableofcontents
\pagebreak

\section{Introduction}
Durant notre 4\up{ème} année d'études à l'INSA de Rennes, nous devons réaliser un projet s'étalant sur l'ensemble de l'année. Notre groupe a choisi de travailler sur le projet \enquote{Avalon}, qui consiste en la réalisation d'un logiciel permettant de simuler un appartement dans lequel les utilisateurs sont amenés à utiliser des équipements domotiques.

Nous commencerons par détailler le contexte du projet, en présentant la réalité virtuelle et le centre de Kerpape. Ensuite, nous proposerons un cahier des charges associé aux solutions techniques retenues pour implémenter les divers scénarios.

Nous concluerons avec une brève présentation des logiciels et matériels utilisables ainsi qu'une planification prévisionnelle du projet.

\section{Contexte}


%Introduction plus générale, dans le monde de l’informatique, pas que le projet, définitions associées (RV, interactions, immersion, domotique)
%Qqes éléments sur l’impact sociétal de l’aide à la personne 
%Présentation sujet (fait le lien), partenariat INSA/Kerpape/(3è partenaire) ; mettre en avant le fait que demandeur extérieur 
	La notion de réalité virtuelle, contrairement à ce que l'on pourrait penser, n'est pas récente, et date du début des années 80 quand Jaron Lanier, un informaticien américain pionnier du domaine, l'a popularisée. Cette notion n'a pourtant pas, à l'origine, l'exact sens qu'on lui prête habituellement aujourd'hui : une réalité virtuelle sous-entendrait qu'il s'agit d'une copie exacte de la réalité, ce qui n'est jamais le cas faute de moyens techniques, et n'est pas toujours recherché. Le terme venant  de l'anglais, \emph{virtual} peut se traduire par \emph{virtuelle} mais aussi par \emph{quasi} ou \emph{pratiquement}, or cette notion de \emph{quasi-réalité} correspondrait mieux à ce qu'est effectivement la réalité virtuelle. 
\\

En effet, de manière plus formelle, on peut définir la réalité virtuelle comme suit :\begin{quote}
%« 
La réalité virtuelle est un domaine scientifique et technique exploitant l'informatique et les interfaces comportementales en vue de simuler dans un monde virtuel le comportement d'entités 3D, qui sont en interaction en temps réel entre elles et avec un ou des utilisateurs en immersion pseudo-naturelles par l'intermédiaire de canaux sensori-moteurs.
% »
\end{quote}
%Il faudra penser à citer la source : Le traité de la RV
Cela signifie que pour que l'on puisse parler de réalité virtuelle il faut que plusieurs conditions soient réunies :	
\begin{itemize}
\item Les interfaces comportementales désignent les interfaces entre l'utilisateur et le monde virtuel. Il existe dans un premier temps des interfaces motrices, qui permettent de reconnaître les différentes actions que l'utilisateur peut entreprendre (mouvements, voix, etc) pour que le système informatique gérant le monde virtuel puisse les prendre en compte. Les interfaces sensorielles ensuite font la liaison dans l'autre direction et informent l'utilisateur de l'état du monde virtuel et des modifications éventuelles que ses actions ont entraîné (images, sons, etc). Ces deux types d'interface permettent de gérer l'interaction avec le monde virtuel et sont donc nécessaires pour parler de réalité virtuelle. 
\item L'immersion caractérise le fait que les interfaces entre l'utilisateur et le monde virtuel se fassent oublier, et ce, en ressemblant le plus possible aux méthodes d'interaction que nous avons avec le monde réel. Elle ne peut pas être parfaite, car il y a toujours du matériel à utiliser qui n'est pas nécessaire pour interagir avec le monde réel (lunettes 3D, WiiMote, etc) mais doit être la plus totale possible.
\end{itemize}


La notion de réalité virtuelle implique donc également celles d'immersion et d'interaction, elles sont en fait constitutives de ce  qu'est la réalité virtuelle. L'utilisateur doit pouvoir faire des actions motrices sur son environnement, il doit pouvoir agir dessus que ce soit par les mouvements, la parole, les gestes... Il n'y a pas de règle définie tant qu'il s'agit d'une activité motrice. 
Une activité sensorielle, par ailleurs, signifie que l'utilisateur percevra un impact de ses actions sur le monde virtuel. Encore une fois il n'y a pas de liste de réponses sensorielles acceptables, ce peut être un son, une modification de l'affichage...	%Eventuellement : le graphe p. 9 du Traité
\pagebreak

\section{Cahier des charges}
\subsection{Partie fonctionnelle}
  
Nous avons eu l’occasion d’échanger avec les membres de Kerpape, ce qui nous a permis de définir le cahier des charges de l’application.\newline
En effet, l'utilisation des appartements tremplins qui permet la mise en situation des patients dans un environnement réel, rencontre des difficultés avec les patients ayant des troubles cognitifs. La domotique trés présente dans l'appartement rencontre donc des difficultés de compréhension et d’utilisation de la part des patients, d'où l'idée d'un environement virtuel d'apprentissage en amont.

\subsubsection{Modes de fonctionnement}

Le programme doit comporter trois modes d’utilisation dont deux assistent en partie l’utilisateur pour son apprentissage. Le troisième permet une interaction avec l'environnement de manière plus autonome.
\newline 

\textbf{Apprentissage symbolique}
\newline 

Ce mode doit permettre de tout apprendre depuis le début et de comprendre le fonctionnement global des différentes situations auxquelles l'utilisateur peut être confronté. Il comporte des vues statiques, “zoomées”, avec la mise en évidence d’action à réaliser, ainsi que des indications visuelles symboliques.
\newline 

\textbf{Assisté}
\newline 

Dans un environnement réaliste, le logiciel donne des indications légères pour permettre de retrouver les actions à faire. Ces indications sont activables par les ergothérapeutes. \newline 
Par exemple :\newline 
- la surbrillance des objets à actionner,\newline 
- lors de l'activation d'une action on accède à une vue fixe avec les états courants des équipements afin de faciliter la compréhension action/objet pour l'utilisateur.
\newline 

\textbf{Autonome}
\newline 

L’utilisateur ne reçoit plus d’information ou d’indication pour effectuer son parcours, il est dans le décor le plus réaliste possible, pour valider son autonomie. Il doit alors actionner les différents objets et se rendre compte par lui même (déplacement) des actions qu'il a éffectué.

\subsubsection{Points de vue : endocentré, exocentré}

Deux points de vue sont configurables. Une vue à la troisième personne (exocentrée), et une vue à la première personne (endocentrée).

\subsubsection{Déplacements, mise en situation et interactions}

L'utilisateur peut se déplacer librement dans l'appartement en mode autonome ; en parallèle il peut choisir de se mettre en situation sur les différents scénarios proposé afin d'interragir avec les différents objets prévus pour le scénario. 

Centrées autour d’un bloc d’interrupteurs, les interactions comprennent notamment de pouvoir ouvrir/fermer les portes (porte du hall avec fermeture automatique / porte de l’appartement avec fermeture volontaire), d'allumer/éteindre les lumières (commande variateur / commande ON/OFF) et de monter/descendre les volets.

\subsubsection{Scénarios}
Trois scénarios autour de l'appel sur le téléphone sont à prévoir avec des actions différentes à entreprendre décrites dans l'étude fonctionnelle. 
\newline 

\textbf{Appel téléphonique: }\textit{Appel téléphonique (d’un proche ou d’une personne qui se serait trompée de numéro). }\newline 
%- L'utilisateur doit pouvoir décrocher le téléphone pour entrer en communication puis raccrocher quand la communication est terminée.
%\newline 

\textbf{Interphone infirmier: } \textit{Appel venant du portier audio/vidéo sur le téléphone (d’un infirmier qui souhaiterait entrer). }\newline 
%L'utilisateur doit pouvoir décrocher le téléphone, communiquer avec l'infirmier, raccrocher le téléphone et ouvrir la porte.
%\newline 

\textbf{Interphone inconnu: } \textit{Appel venant du portier audio/vidéo sur le téléphone (d’un inconnu). }\newline 
%L'utilisateur doit pouvoir décrocher le téléphone pour entrer en communication, allumer la TV pour voir la vidéo puis éteindre la TV et raccrocher le téléphone à la fin de la conversation.



 

\pagebreak
\section{Etude fonctionnelle}

La solution logicielle que nous allons implémenter doit intégrer plusieurs scénarios de fonctionnement. Ceux-ci permettront de tester les aptitudes du patient en rééducation dans un contexte banal. Ces scénarios sont axés sur l'interaction entre le patient et le monde extérieur, via le \enquote{Domophone}.

\subsection{Scénario 1: Appel téléphonique entrant}

Le premier scénario à implémenter est assez basique. Il s'agit du cas où le résident reçoit un appel téléphonique. Il doit alors décrocher puis raccrocher le téléphone au terme de la conversation.

D'un point de vue logiciel, il s'agira de déclencher la sonnerie du téléphone pour avertir l'utilisateur. Une icône de téléphone pourra être affichée en bas de l'écran, dans un coin. Ensuite, l'utilisateur devra se déplacer vers la zone du téléphone. Suivant le mode d'utilisation, le rendu sera différent : En mode symbolique, le logiciel basculera sur une vue du téléphone et donnera les instructions pour décrocher puis pour raccrocher. En mode assisté, le téléphone (ou la télécommande universelle) sera en surbrillance et si l'utilisateur met trop de temps pour agir, certains boutons peuvent se mettre en surbrillance ou clignoter.

\subsection{Scénario 2: Visite de l'infirmier}

Ce scénario correspond à la visite d'une personne connue par le résident. Lorsque que le visiteur arrive, il utilise l'interphone pour demander l'ouverture de la porte extérieure. Le patient en rééducation doit décrocher le domophone, activer l'ouverture de la porte et raccrocher. L'ouverture de la porte peut se faire par un code entré dans le domophone, un appui sur la télécommande universelle ou sur l'interrupteur à l'entrée. Ce cas correspond bien à la visite d'un infirmier ou d'une quelconque personne que le résident peut reconnaître par la voix.

Pour cette implémentation la différence avec le scénario 1 portera sur l'ouverture de la porte extérieure, soit via la télécommande, le domophone ou l'interrupteur. En mode symbolique, le basculement entre les vues sera automatique. Le mode assisté mettra en évidence l'ensemble des interrupteurs et boutons de commandes, puis en cas d'hésitation, le bon interrupteur sera mis en valeur.

\subsection{Scénario 3: Visite d'un inconnu}

Le scénario le plus complet correspond au passage d'un visiteur inconnu. En effet, le résident devra dans cette situation décrocher le téléphone, puis activer l'affichage vidéo de l'interphone sur l'écran de télévision de l'appartement. Pour cela il devra éventuellement allumer sa télévision et passer au canal 80. Ensuite, il devra vérifier qu'il peut laisser entrer le visiteur et le cas échéant, lui ouvrir le portail, puis raccrocher. S'il ne désire pas lui ouvrir, il lui faudra juste raccrocher.
Ce scénario convient à plusieurs situations courantes : visite d'un réparateur, d'un livreur, d'un témoin de Jéhovah ...

Pour réaliser la partie \enquote{visio TV} de ce scénario l'utilisateur devra activer la vision via le domophone, allumer la télévision et éteindre la télévision. Là encore, les 3 modes de fonctionnement vont modifier le comportement du logiciel et orienter plus ou moins l'utilisateur vers la télévision. En mode symbolique, l'utilisateur sera guidé pas à pas pour :
\begin{itemize}
	\item activer la visio depuis le domophone ;
	\item allumer la télévision ;
	\item éteindre la télévision.
\end{itemize}
La suite du scénario reprend le fonctionnement final du scénario 2, qui lui-même reprend la fin du scénario 1.

\subsection{Cas d'utilisation}
\begin{figure}[h]
  \caption{Diagramme des cas d'utilisation}
  %\centering
  \includegraphics[width=\textwidth]{1-PreEtude/img/diagramme}
\end{figure}

\pagebreak
\input{1-PreEtude/etude-technique.tex}
\pagebreak
\section{Spécifications}
	%@author : T'es pas à la maison mon gars , parle bien !
	%Le "aka so mega casse-couilles" etait une demande de mon voisin, je suis pas responsable :D En bref, un oubli.

\subsection{Données}
	Pour ce projet, nous avons à notre disposition une modélisation 3D de l'appartement tremplin. Cette modélisation, fournie par Kerpape, est au format .max. L'un de notre premier travail sera donc d'ouvrir ce fichier, et de le convertir au format natif d'Unity. Unity dispose bien sur de fonctionnalités d'import dans d'autres formats, mais Unity est connu pour être un peu capricieux dans ses interactions avec les .max.

	Les textures sont parfois mal gérées ou absentes, il faudra donc trouver un moyen d'interfacer correctement et efficacement la ressource avec Unity, soit avec une conversion de format de fichier, soit en réintégrant dans Unity ce qui aura été perdu à l'import.

\subsection{Logiciels}
	Au cours de ce projet, nous allons devoir travailler avec plusieurs environnements logiciels : les modeleurs 3D et les environnements de développement. La première catégorie de logiciels permet de modifier le modèle ou de créer de nouveaux éléments, comme un téléphone ; tandis que la seconde permet d'écrire la logique pour réaliser la partie interactive du projet.

	Voici les solutions logicielles que nous allons utiliser pour la réalisation :
	\subsubsection{3ds Max}
		\noindent\begin{minipage}{0.3\textwidth}
			\includegraphics[width=\linewidth]{1-PreEtude/img/3dsmax_logo}
			\end{minipage}
			\hfill
			\begin{minipage}{0.65\textwidth}
			3ds Max\cite{3dsmax}, le célèbre modeleur 3D d'Autodesk n'est plus à présenter; aujourd'hui encore considéré comme la référence en matière de modélsation 3D, et ce depuis plus de 10 ans, il est grandement utilisé dans l'industrie vidéoludique et filmographique.
			Ce logiciel est l'évolution de 3D studio, sorti sous DOS en 1990. 3ds Max lui a succédé en 1996 et dispose de nouvelles versions stables tout les 6 mois, la dernière étant la 2015 SP2 sortie le 20 mars 2014.
			Ce logiciel est cependant vendu à un prix élevé et n'est disponible que sur Windows, ce pourquoi nous utiliserons d'autres logiciels en parallèle.
		\end{minipage}


	\subsubsection{Blender}
		\noindent\begin{minipage}{0.3\textwidth}
			\includegraphics[width=\linewidth]{1-PreEtude/img/blender_logo}
			\end{minipage}
			\hfill
			\begin{minipage}{0.65\textwidth}
			Blender\cite{blender} est le modeleur 3D libre le plus avancé. Le projet Blender fut lancé en 1995 et était initialement un logiciel propriétaire developpé en interne par Neo Geo et Not a Number Technologies.
			Son code source fut ouvert au public le 17 septembre 2002 après une campagne de dons. Aujourd'hui distribué sous licence GNU/GPLv2, il est soutenue par la Blender Foundation. Celle ci a réalisé quelques films d'animations tels que Big Buck Bunny ou Sintel pour promouvoir le logiciel.
			Le projet est régulièrement mis à jour, la dernière version (2.72) à l'heure actuelle datant du 14 octobre 2014.
			Blender est disponible sous Windows, Mac, Linux et BSD et inclut toutes les fonctionnalités classiques d'un logiciel de se type, mais inclu aussi des fonctionnalités d'extension via des scripts Python ainsi qu'un moteur de jeu.
			Ces possibilités supplémentaires ne seront cependant pas utilisés dans ce projet, car elles ne répondent que partiellement à notre problématique.
		\end{minipage}


	\subsubsection{Unity}
		\noindent\begin{minipage}{0.3\textwidth}
			\includegraphics[width=\linewidth]{1-PreEtude/img/unity_logo}
			\end{minipage}
			\hfill
			\begin{minipage}{0.65\textwidth}
			Unity\cite{unity} est un environnement et un moteur de rendu 3D ; initialement pensé pour les jeux vidéos, il permet de créer ou modifier des environnements en 3D et correspond parfaitement à nos besoins pour la modification du modèle d'appartement tremplin.
			De plus, contrairement à d'autres moteurs 3D, Unity propose une version gratuite, qui regroupe la majorité des fonctionnalités de la version payante à l'exception de la compilation 64 bits, la gestion des ombres ainsi que la diffusion commerciale.
			Il permet de réaliser toutes les actions classiques d'un logiciel de ce type, comme construire des objets, les animer, interagir avec, etc. Toutes ces actions sont effectuées de manière native depuis l'interface en écrivant des scripts en C\#, Javascript ou Boo via l'API fournie.
			Le développement y est grandement facilité, car celle-ci inclus de nombreuses fonctionnalités. Il n'y a donc pas besoin de sortir du logiciel pour développer.
			Unity est actuellement utilisé dans la salle Immersia de l'Irisa.
		\end{minipage}


	\subsubsection{VRPN}

		VRPN\cite{vrpn} (Virtual-Reality Peripheral Network) est un système de gestion de périphériques pour la réalité virtuelle. Cette bibliothèque offre une interface entre le matériel et l'application. Elle offre des classes génériques pour chaque type de périphériques; par exemple, tout les trackers sont gérés de la même façon.
		Le projet fut initié en 1998 par Russell M. Taylor II de l'université de Caroline du Nord et est aujourd'hui maintenu par une vingtaine de contributeurs.

	\subsubsection{MiddleVR}
		\noindent\begin{minipage}{0.3\textwidth}
			\includegraphics[width=\linewidth]{1-PreEtude/img/middlevr_logo}
			\end{minipage}
			\hfill
			\begin{minipage}{0.65\textwidth}
			MiddleVR\cite{middlevr} est un plugin compatible avec Unity qui s'appuie sur  VRPN. Développé par I'm in VR (PARIS, France) , il permet de gérer les interactions entre l'utilisateur et son environnement et est spécialement conçu pour les environnements en réalité virtuelle.
			L'objectif de MiddleVR est de pouvoir programmer l'application en s'abstrayant des spécificités matérielles pour qu'elle soit déployable partout.
			Il propose une couche d'abstraction entre les périphériques et Unity. Ces périphériques comprennent ceux d'entrée (de capture), tels que les classiques claviers et souris mais aussi les bras à retour de force, des trackers ou des MS Kinect, ainsi que ceux de sortie (de restitution), comme les écrans, les vidéoprojecteurs 3D mais aussi le son ou le retour de force des bras.
			MiddleVR gère nativement la stéréoscopie active ou passive, c'est donc un bon add-on pour s'abstraire de l'aspect interface homme-machine et capteurs.
		\end{minipage}


	\subsubsection{\#Five}
		\#Five est une bibliothèque pour Unity developpée en interne au sein de l'Irisa. \#Five ajoute plusieurs couches, en particulier une couche de gestion de relations entre les objets (Une interaction avec un objet A effectue une action sur un objet B).
		Ces relations permettent des scénarios de haut niveau, avec un ordre relatif des interactions entre les étapes. Par exemple, pour démonter une culasse, il faut dévisser les boulons; l'ordre dans lequel on dévisse lesdits boulons importe peu.
		\#Five inclut aussi une infrastructure gérant le travail collaboratif sur environnement virtuel, c'est-à-dire quand plusieurs personnes interagissent sur la même scène.
		Enfin, il permet la gestion des humains virtuels, avec qui il est possible de collaborer.
		\#Five ne sera pas utilisé dans un premier temps dans ce projet, mais sera potentiellement intégré à notre application.


\subsection{Matériels et environnement technique}

Ce projet consistant en l'utilisation d'un environnement 3D virtuel (appartement de Kerpape aidant à la réhabilitation de personnes lourdement handicapées), où l'utilisateur sera amené à avoir des interactions avec cet environnement, nous allons donc avoir accès à la salle de réalité virtuelle $\mu$RV de l'INSA Rennes et à son matériel. \`A nous d'utiliser ce dernier à bon escient, pour répondre au mieux à la demande de Kerpape et pour pouvoir proposer un environnement d'apprentissage le plus performant possible notamment au niveau des interactions avec l'utilisateur.

\subsubsection{Matériel d'immersion}
Nous disposons de différents outils pour immerger l'utilisateur au coeur de l'environnement virtuel :
\\

\textbf{Lunettes nVidia 3D Vision et Récepteur 3D Vision}
\\

Ces lunettes, sur batterie, permettent de visualiser une stéréoscopie active. Elles sont reconnues par l'ordinateur grâce à une base qui émet des signaux infrarouges.
\\

\textbf{Vidéoprojecteur 3D}
\\

Le vidéoprojecteur permet d'avoir un écran 3D à disposition pour s'immerger dans l'environnement plus facilement et à plusieurs avec une taille d'image bien supérieure à celle d'un écran d'ordinateur classique.
\\


\textbf{Oculus Rift}
\\

L'appareil se présente sous la forme d'un masque recouvrant les yeux et attaché au visage par une sangle fermée à l'arrière du crâne. Un écran plat numérique est placé à quelques centimètres en face des yeux, perpendiculairement à l'axe du regard. Cet écran affiche une image stéréoscopique déformée numériquement pour inverser la distorsion optique créée par deux lentilles situées en face de chaque œil. Divers capteurs permettent de détecter les mouvements de tête de l'utilisateur, ce qui permet d'adapter en temps réel l'image projetée sur l'écran, afin de produire l'illusion d'une immersion dans la scène restituée.
Le dispositif se démarque des systèmes comparables expérimentés précédemment par la très courte latence dans le suivi des mouvements de la tête, une coupure totale avec le monde exétérieur et par l'important champ de vision offert (360°).
\\

\textbf{Plate-forme Immersia}
\\

Située à l'Irisa, en forme de « L », la salle Immersia est dotée d'un équipement immersif plongeant l'utilisateur dans un monde visuel et auditif de haute qualité. C'est également la plus grande salle d'Europe.
Elle est constituée  :
\begin{itemize}
  \item d'un système visuel utilisant 11 projecteurs : 8 Barco Galaxy NW12 et 3 Barco Galaxy 7+ ;
  \item d'un écran de verre de 9,60 mètres de long où sont projetées par l'arrière, les images stéréoscopiques ;
  \item d'un système de localisation ART permettant à des objets réels d'être localisés à l'intérieur de la plate-forme ;
  \item d'un système de rendu sonore fourni par un processeur Yamaha, lié soit à des hauts-parleurs Genelec au format sonore 10.2, soit à des casques Beyer Dynamic avec un format sonore virtuel de 5.1, contrôlé par la position de l'utilisateur.
\end{itemize}

\subsubsection{Matériel d'interaction}
Une fois l'utilisateur intégré dans la modélisation virtuelle de l'appartement, nous avons à notre disposition plusieurs équipements pour le faire interagir avec son environnement :
\\

\textbf{Microsoft Kinect}
\\
La Kinect développée par Microsoft, est le périphérique permettant de se rapprocher le plus de ce que l'on peut imaginer en réalité virtuelle. En effet, cela permet à l'utilisateur d'interagir en totale immersion avec l'environnement sans aucun support matériel à manipuler. Elle permet de détecter la présence de 6 personnes et de suivre les mouvements de deux utilisateurs actifs grâce à ses lentilles placées sur un socle motorisé. Sa portée est comprise entre 1,2m et 3,5m. L'intérêt principal de la Kinect est que l'utilisateur puisse se déplacer librement dans une pièce sans avoir à manipuler une quelconque manette.
\\

\textbf{WiiMote, Nunchuck et WiiMotionPlusInside}
\\
La Wiimote est la manette fournie avec la console de jeu Wii, développée par Nintendo. Elle est composée de deux parties : la manette principale et le Nunchuk. Bien que moins pratique que la Kinect (l'utilisateur est contraint de manipuler une manette), la Wiimote dispose d'un système de liaison bluetooth d'une portée de 10m permettant une utilisation sans fil. L'avantage de la Wiimote réside dans le nombre et la précision des informations qu'elle est capable de fournir au programme.
La Wiimote est une manette précise disposant de multiples boutons pouvant permettre à l'utilisateur de marcher, attraper, accéder au menu.
\newline
Elle permet en effet de :
\begin{itemize}
  \item mesurer les accélérations selon 3 axes (axes naturels 3D) grâces aux accéléromètres placés dans la manette principale et le Nunchuk ;
  \item mesurer l'inclinaison de la manette principale selon les 3 axes naturels ;
  \item mesurer la distance ainsi que la position entre la Wiimote et la barre infrarouge (référentiel). \\
\end{itemize}

\textbf{Joystick Extreme 3D Pro Logitech}
\\
Avec ses commandes avancées et sa gouverne à manche rotatif, ce joystick est prévu pour être connecté à un ordinateur et est utilisé pour des jeux de combat aérien acrobatique. Il comporte également de nombreux boutons.
\\

\textbf{Bras à retour de force Novint Falcon}
\\
Le Falcon, de la société Novint, est un périphérique haptique branché en USB. Il permet de ressentir le retour d'effort, et donc la texture et la résistance des objets, leur poids... Une boule est reliée au support par des tiges. C'est elle que l'on déplace et qui transmet à la main de l'utilisateur les efforts des moteurs.
\\

\subsection{Techniques d'interaction}
Au vu de toutes les ressources technologiques dont nous disposons, nous avons décidé de privilégier les techniques d'interaction suivantes :

\subsubsection{Interface Clavier/Souris et écran d'ordinateur}
Une première version utilisable sur un ordinateur avec ses périphériques de base (clavier et souris) permettant d'avoir un environnement d'apprentissage fonctionnel et testable rapidement, et également très portable.

\subsubsection{Compatibilité avec tous les périphériques via MiddleVR}
L'objectif serait de présenter une application fonctionnant avec tous les périphériques à notre disposition et cités précédemment, grâce à une reconnaissance et une configuration automatique des périphériques.
\\

\pagebreak
\section{Planification}

L'un des buts de ce rapport est de prevoir l'organisation de notre travail ainsi que de faire une estimation du planning de nos productions.

\subsection{Versions intermédiaires}
Lors de la réalisation de ce projet, nous allons produire plusieurs versions intermédiaires pour nous permettre de construire chaque fonctionnalité au fur et à mesure :
\begin{itemize}
  \item Version \textnumero1 : Implémentation de l'appartement complet dans un environement 3d. Le personnage peut se déplacer librement à l'intérieur, mais sans collisions avec l'environnement.
  \item Version \textnumero2 : Une deuxième version ajoute les interactions basiques : collision avec les obstacles, possibilité d'utiliser les interrupteurs. Elle correspond au mode \enquote{Utilisation} du logiciel.
  \item Version \textnumero3 : La troisième version permet d'utiliser le mode \enquote{Apprentissage}.
  \item Version \textnumero4 : La quatrième version implémente les différents scénarios d'utilisation du logiciel et fonctionne avec PC+clavier/souris.
  \item Version \textnumero5 : La version finale fonctionne en réalité virtuelle dans ces 3 envionnements : salle $\mu$RV, salle Immersia, casque de réalité virtuelle.
\end{itemize}

\subsection{Organisation du travail}
Pour ce projet, nous avons mis en place un certain nombre d'outils de travail collaboratif :

\begin{itemize}
  \item GitHub : pour nos documents versionnés tel que le code source ;
  \item Cloud privé : pour les documents confidentiels ;
  \item Google Drive : pour tout autre document ;
  \item Wiki : permet une base centrée de connaissance sur la salle $\mu$RV et de tous les projets qui y ont été réalisés.
\end{itemize}


\subsection{Estimation du planning}
Au cours de l'année, nous sommes chargés de rédiger 6 documents (rapports, documentation) et de réaliser un certain nombre de versions intermédiaires de notre logiciel.
Voici une estimation du planning pour l'année :

\begin{tabular}{|l|l|}
\hline
  Date &
  Production \\
\hline
  19 novembre &
  Rapport de spécification fonctionnelle VP \\
\hline
  27 novembre &
  Rapport de spécification fonctionnelle VF \\
\hline
  9 décembre &
  Version PC \textnumero1 \\
\hline
  9 décembre &
  Dossier de planification Initial VP \\
\hline
  17 décembre &
  Dossier de planification Initial VF \\
\hline
  9 janvier &
  Version PC \textnumero2 \\
\hline
  6 février &
  Rapport de conception logicielle VP \\
\hline
  9 février &
  Version PC \textnumero3 \\
\hline
  12 février &
  Rapport de conception logicielle VF \\
\hline
  9 mars &
  Version PC \textnumero4 \\
\hline
  25 mars &
  Page HTML VP \\
\hline
  2 avril &
  Page HTML VF \\
\hline
  9 avril &
  Version PC \textnumero5 \\
\hline
  16 mai &
  Documentation en Ligne VP \\
\hline
  21 mai &
  Rapport Final/Annexes + Bilan Planification \\
\hline
  26 mai &
  Rapport Final/Annexes + Bilan Planification \\
\hline
  28 mai &
  Documentation en Ligne VF \\
\hline
\end{tabular}


\pagebreak

\section{Conclusion}
Au cours de cette étude, nous avons eu l'occasion de rencontrer nos interlocuteurs de Kerpape et nous avons pu dresser un premier portrait de la solution technique que nous allons développer.
La prochaine étape va être de mettre en application les solutions évoquées, pour avoir une première version du logiciel permettant de se déplacer librement dans l'appartement.
Il faut donc importer le modèle dans une scène sous Unity, et y rajouter les lumières ainsi que les éléments manquants (panneau de commandes, téléphone).
Nous redécouperons de manière plus détaillée les tâches à réaliser dans le prochain rapport où nous planifierons l'ensemble de la réalisation technique.

\pagebreak

\bibliography{1-PreEtude/biblio}

\end{document}
