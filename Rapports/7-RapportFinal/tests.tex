\section{ANNEXE I : Tests}

\subsection{Tests unitaires}
L'architecture de notre projet, qui utilise le moteur de jeu Unity3D, ne nous a pas permis d'en faire ! \newline

Les tests unitaires, le type de tests que nous avons le plus utilisés au long de notre formation, fonctionnent très bien avec des projets ne comportant que du code informatique, car ainsi nous aurions pu tester chaque bloc du projet indépendament avant de les assembler entre eux. \newline

Dans le cas de Unity, l'écriture de code n'intervient que dans les scripts en C\# qui ne représentent pas l'intégralité du projet. De plus, leur utilisation étant intimimement mêlée aux actions de l'utilisateur dans l'univers 3D, il était impossible de les tester indépendament. \newline

Unity propose tout de même un ensemble d'outils de test, qui permet entre autres choses de réaliser des tests unitaires, mais les délais du projet ne nous ont pas permis de prendre en main cesoutils. Les tests que nous avons réalisé au long de notre projet sont donc principalement des tests fonctionnels et d'acceptation. 

\subsection{Tests fonctionnels}

Les tests fonctionnels représentent la majorité de ceux que nous avons menés au cours du projet Avalon. Faciles à mettre en place, ils nous assuraient qu'une modification n'entraîne pas une régression nattendue. Nous vérifiions chaque fonctionnalité indépendament, sans prendre la peine de mettre en place une scénario d'utilisation réaliste mais simplement en vérifiant que le programme répondait de la manière souhaitée à une situation donnée (conformité aux spécifications du projet).\newline

Chaque fonctionnalité, au moment de son intégration, a été dûment testée, ainsi qu'à plusieurs reprises par la suite à chaque fois qu'une modification était suceptible de l'impacter. À titre d'exemple, nous avons testé les déplacements au clavier/souris quand nous avons créé le script utilisé, puis plus tard quand nous avons modifié la manière de contrôler la wand qui fait office de curseur à l'écran et requérait une attention particulière, ou encore quand nous avons créé une zone où l'utilisateur perdait le contrôle de ses déplacements et était déplacé « sur des rails ».\newline

L'avantage principal de tests est leur facilité, et ils nous permettaient de tester chaque fonctionnalité indépendament et en temps réel, mais en contrepartie nous courrions le risque de ne pas remarquer cetains cas d'erreurs qui ne seraien pas apparus clairement, d'une part parce qu'étant les concepteurs des fonctionnalités en question nous avions un \textit{a priori} quant à la façon dont il fallait s'en servir, et d'autre part car nous n'avions pas le temps de réfléchir à tous les cas exotiques qui pourraient survenir. 

\subsection{Tests d'acceptation}

Les tests d'acceptation représentent la seconde catégorie de tests que nous avons appliqués. Moins fréquents que les tests fonctionnels, ils sont plus intensifs. \newline

Les tests d'acceptation vérifient le fonctionnement en interaction d'un grand nombre de \textit{features} et sanctionnent une avancée significative du projet. Il s'agit généralement de tester un scénario d'utilisation réaliste qu'un usager serait suceptible de réaliser par lui-même sans connaître les détils d'implémentation du projet. \newline

Pour repredre les exemples du point précédent, une fois toutes les fonctionnalités mentionnées disponibles, nous avons essayé d'apparaître dans la scène, se déplacer librement quelques instants, puis se rendre dans la zone où l'on perdait le contrôle du personnage et en repartir, le tout sans qu'il n'y ait de bug de collision (e.g. traverser un mur ou se glisser dans l'interstice d'une porte). 