\section{Evolution du projet}
Lorque le centre Kerpape nous a proposé ce projet, nous avons établi une liste de fonctionnalités à implémenter,

\subsection{Objectif initial}
L'objectif du projet est de développer une application pour les patients du centre Kerpape. Ce centre possède des résidences, appelés \enquote{appartements tremplins} équipés d'appareils domotiques pour que des personnes handicapées puissent vivre de manière indépendantes. Le but de l'application est de permettre à ces patients de découvrir et de s'habituer aux différents équipement domotiques. Ils doivent donc être capables d'interagir avec ces élements dans la scène 3D à travers des périphériques tels que le clavier et la souris.

\subsection{Résultat final}
L'application finale permet de réaliser la plupart des actions décrites dans le cahier des charges : L'utilisateur peut se déplacer librement dans l'appartement tout en regardant tout autour de lui. Il peut interagir avec certains éléments tels que les interrupteurs ou le domophone pour allumer la lumière, ouvrir la porte ou les volets, et répondre au téléphone. L'utilisateur était capable de sauter, mais cette fonction a été désactivée car elle n'est pas réaliste pour une personne a mobilité réduite.

Nous avons également implémenté différents modes d'apprentissage pour assister l'utilisateur dans sa découverte de l'appartement. Dans le mode autonome, tout se passe comme dans le monde réel, si l'utilisateur appuie sur l'interrupteur relié à la lumière de la cuisine, cette lumière va s'allumer. Dans le mode assisté, un écran va s'afficher en bas à droite de l'écran et indiquer quel élément de la scène

\subsection{Améliorations possibles}

Bien que notre application remplisse les fonctions principales demandées, nous avons envisagé avec Kerpape d'autres fonctionnalitées pour la compléter.
\begin{itemize}
\item Bien que nous implémentons déjà un certain nombre de scénarios, il ne pourrait être que positif d'en ajouter pour que l'utilisateur se familiarise d'avantage avec des situations de la vie courante.
\item Les appartements possèdent une télécommande qui peut agir sur la plupart des appareils, et il serait intéressant de l'implémenter. Une représentation de cette télécommande a déjà été ajoutée dans la scène Unity3D, il reste juste à insérer des icones sur ses boutons, et à leur attribuer des actions.
\item Nos clients de Kerpape nous ont suggérés que les thérapeutes puissent personnaliser certaines interfaces en fonction du handicap du patient. Par exemple, modifier la télécommande si le patient ne peut pas se servir de ses doigts.
\item On nous a également proposé d'ajouter une base de donnée sur les patients. Elle contiendrait des fiches d'informations que le thérapeute pourrait lire et modifier lorsqu'il travail plusieurs fois avec le même patient.
\end{itemize}
