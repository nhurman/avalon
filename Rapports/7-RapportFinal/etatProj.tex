\section{Realisation du cahier des charges}
Nous avons eu l'occasion d'échanger avec les membres de Kerpape, ce qui nous a permis de définir le cahier des charges de l'application, que nous allons détailler dans cette partie.
\begin{tabular}{|p{12cm}|l|}
	\hline
	Tache & Etat \\ \hline
	Afficher l'appartement en 3D & Realise \\ \hline
	Deplacement dans l'appartement & Realise \\ \hline
	Vue exocentrée : Une vue à la troisième personne permet de voir le déplacement de l'utilisateur dans la scène. & Realise \\ \hline
	Vue endocentrée : Une vue à la première personne permet de voir par les yeux de l'avatar  virtuel. & Realise \\ \hline
	
	Implémenter différents scénario \\ \hline
	Possibilité de basculer entre les modes en cours de jeu & PIKACHU \\ \hline
	Mode symbolique : mise en évidence d'actions à réaliser, ainsi que des indications visuelles symboliques. & Realise \\ \hline
	Mode assisté : lors de l'activation d'une action on accède à une vue fixe avec les états courants des équipements afin de faciliter la compréhension du lien action/objet pour l'utilisateur & Realise \\ \hline
	Mode autonome : L'utilisateur ne reçoit plus d'information ou d'indication pour effectuer son parcours, il est dans le décor le plus réaliste possible, pour valider son autonomie. Il doit alors actionner les différents objets et se rendre compte par lui même (déplacement) des actions qu'il a effectuées.  & Realise \\ \hline
	
	Scenarios :  L'utilisateur peut choisir de se mettre en situation sur les différents scénarios proposés afin d'interagir avec les différents objets prévus pour le scénario.\\ \hline
	Interractions : L'utilisateur doit pouvoir interagir avec son environnement (portes, lumières, volets). & Réalisé \\ \hline
	Scenario 1 - Appel téléphonique : Appel téléphonique (d'un proche ou d'une personne qui se serait trompée de numéro). L'utilisateur doit pouvoir décrocher le téléphone pour entrer en communication puis raccrocher quand la communication est terminée. & realise \\ \hline
	Scenario 2 - Interphone infirmier : Appel venant du portier audio/vidéo sur le téléphone (d'un infirmier qui souhaiterait entrer). L'utilisateur doit pouvoir décrocher le téléphone, communiquer avec l'infirmier, raccrocher le téléphone et ouvrir la porte.& realise \\ \hline
	Scenario 3 - Interphone inconnu : Appel venant du portier audio/vidéo sur le téléphone (d'un inconnu). L'utilisateur doit pouvoir décrocher le téléphone pour entreren communication, allumer la TV pour voir la vidéo puis éteindre la TV et raccrocher le téléphone à la fin de la conversation.& realise \\ \hline
	
	Application portable - faire une application aisément portable sur différents systèmes d'utilisation & Windows seulement \\ \hline
	Capable de fonctionner avec de nombreux périphériques différents & MiddleVR \\ \hline
	Salle immersive & Testé fonctionnel dans Immersia \\ \hline
	Porter le logiciel sur tablette  & NOPE.JPG \\ \hline
	Casque de réalité virtuelle & ok \\ \hline
	Clavier souris & ok \\ \hline
	
\end{tabular}
