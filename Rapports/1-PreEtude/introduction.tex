\section{Introduction}

% Annonce plan
% Visite kerpape
% Rencontre clients
Notre projet consiste en la modélisation d'appartements tremplins en 3D pleinement interactifs, dans lesquels des personnes handicapées pourront évoluer et apprendre à se servir des différents équipements de domotique présents. 
Le projet est proposé par le centre de rééducation de Kerpape et par l'IRISA, ce qui lui confère un intérêt particulier à nos yeux car le commanditaire est extérieur à l'école, et le projet répond donc au besoin d'un véritable client de la même manière que le ferait un projet rencontré dans notre vie professionnelle. 
Ainsi, nous avons eu l'occasion de nous entretenir avec Willy Allègre et Jean-Paul Departe, ingénieurs de Kerpape à l'origine du projet, tout d'abord lors d'une conférence téléphonique pendant laquelle ils nous ont décrit ce qu'ils attendaient de l'application. Par la suite, un déplacement au centre de Kerpape nous a permis de parfaire l'image que nous nous faisons du résultat attendu et de compléter le cahier des charges de la future application. \newline

Au cours de cette étude de pré-spécifications, nous commencerons par définir plus précisément le contexte de notre projet, en tentant de cerner ce qu'est la réalité virtuelle et son importance dans le domaine de la santé. 
Nous présenterons ensuite le cahier des charges en deux parties, d'abord d'un point de vue fonctionnel, les différentes fonctions qui sont attendues de notre applications, et ensuite d'un point de vue technique, les différentes contraintes que nous aurons à respecter. 
Après cela nous décrirons, dans l'étude fonctionnelle, les différents scénarios types qui nous ont été proposés par Kerpape et auxquels les patients devront pouvoir être confrontés. 
Puis nous spécifierons les différents logiciels à notre disposition ainsi que les différents matériels, et les périphériques que nous prévoyons de rendre utilisables. 
Enfin, nous ébaucherons une planification prévisionnelle du travail à fournir tout au long de l'année. 