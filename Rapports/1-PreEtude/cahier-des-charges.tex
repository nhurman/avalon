
\section{Cahier des charges}
\subsection{Partie fonctionnelle}
  
Nous avons eu l’occasion d’échanger avec les membres de Kerpape, ce qui nous a permis de définir le cahier des charges de l’application.\newline
En effet, l'utilisation des appartements tremplins qui permet la mise en situation des patients dans un environnement réel, rencontre des difficultés avec les patients ayant des troubles cognitifs. La domotique trés présente dans l'appartement rencontre donc des difficultés de compréhension et d’utilisation de la part des patients, d'où l'idée d'un environement virtuel d'apprentissage en amont.

\subsubsection{Modes de fonctionnement}

Le programme doit comporter trois modes d’utilisation dont deux assistent en partie l’utilisateur pour son apprentissage. Le troisième permet une interaction avec l'environnement de manière plus autonome.
\newline 

\textbf{Apprentissage symbolique}
\newline 

Ce mode doit permettre de tout apprendre depuis le début et de comprendre le fonctionnement global des différentes situations auxquelles l'utilisateur peut être confronté. Il comporte des vues statiques, “zoomées”, avec la mise en évidence d’action à réaliser, ainsi que des indications visuelles symboliques.
\newline 

\textbf{Assisté}
\newline 

Dans un environnement réaliste, le logiciel donne des indications légères pour permettre de retrouver les actions à faire. Ces indications sont activables par les ergothérapeutes. \newline 
Par exemple :\newline 
- la surbrillance des objets à actionner,\newline 
- lors de l'activation d'une action on accède à une vue fixe avec les états courants des équipements afin de faciliter la compréhension action/objet pour l'utilisateur.
\newline 

\textbf{Autonome}
\newline 

L’utilisateur ne reçoit plus d’information ou d’indication pour effectuer son parcours, il est dans le décor le plus réaliste possible, pour valider son autonomie. Il doit alors actionner les différents objets et se rendre compte par lui même (déplacement) des actions qu'il a éffectué.

\subsubsection{Points de vue : endocentré, exocentré}

Deux points de vue sont configurables. Une vue à la troisième personne (exocentrée), et une vue à la première personne (endocentrée).

\subsubsection{Déplacements, mise en situation et interactions}

L'utilisateur peut se déplacer librement dans l'appartement en mode autonome ; en parallèle il peut choisir de se mettre en situation sur les différents scénarios proposé afin d'interragir avec les différents objets prévus pour le scénario. 

Centrées autour d’un bloc d’interrupteurs, les interactions comprennent notamment de pouvoir ouvrir/fermer les portes (porte du hall avec fermeture automatique / porte de l’appartement avec fermeture volontaire), d'allumer/éteindre les lumières (commande variateur / commande ON/OFF) et de monter/descendre les volets.

\subsubsection{Scénarios}
Trois scénarios autour de l'appel sur le téléphone sont à prévoir avec des actions différentes à entreprendre décrites dans l'étude fonctionnelle. 
\newline 

\textbf{Appel téléphonique: }\textit{Appel téléphonique (d’un proche ou d’une personne qui se serait trompée de numéro). }\newline 
%- L'utilisateur doit pouvoir décrocher le téléphone pour entrer en communication puis raccrocher quand la communication est terminée.
%\newline 

\textbf{Interphone infirmier: } \textit{Appel venant du portier audio/vidéo sur le téléphone (d’un infirmier qui souhaiterait entrer). }\newline 
%L'utilisateur doit pouvoir décrocher le téléphone, communiquer avec l'infirmier, raccrocher le téléphone et ouvrir la porte.
%\newline 

\textbf{Interphone inconnu: } \textit{Appel venant du portier audio/vidéo sur le téléphone (d’un inconnu). }\newline 
%L'utilisateur doit pouvoir décrocher le téléphone pour entrer en communication, allumer la TV pour voir la vidéo puis éteindre la TV et raccrocher le téléphone à la fin de la conversation.



 
