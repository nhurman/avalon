\section{Sp\'ecifications}

\subsection{Données}
	D'autres travaux on déjà été réalisés : http://infositu.loria.fr/ >> domotique + RV + aide personne
	Pour ce projet, nous avons à notre disposition une modélisation 3D de l'appartement tremplin. Cette modélisation, fournie par Kerpape, est au format .max.
	- format de Unity (fbx)
	Applications de la réalité virtuelle en sciences de la vie :
	Le traitement de troubles de stress post-traumatique (TSPT)
	Le traitement de phobies
	Le traitement d'addictions
	Le contrôle de la douleur des grands brûlés
	Les études comportementales et de perception
	L'éducation à la santé publique
	La formation à l'anatomie, aux pathologies et à la chirurgie pour les étudiants et médecins
	La cyberanatomie
	...et bien d'autres encore.

%	http://fr.wikipedia.org/wiki/Th%C3%A9rapie_par_r%C3%A9alit%C3%A9_virtuelle
%	http://www.fhpmco.fr/2014/08/27/des-lunettes-a-realite-augmentee-pour-aider-les-malvoyants
%	http://www.pourlascience.fr/ewb_pages/a/article-realite-virtuelle-pour-therapies-reelles-21783.php
\subsection{Logiciels}
	Au cours de ce projet, nous allons devoir travailler entre différents environnements logiciels; ceux-ci entrent dans deux grandes catégories : les modeleur 3D et les autres trucs.
	Voici les solutions logicielles que nous allons utiliser pour la réalisation de notre projet :
	
	\subsubsection{3DSmax}
		3DSmax, le célèbre modeleur 3D d’Autodesk n’est plus à présenter; aujourd’hui considéré comme la référence en matière de modélsation 3D, et ce depuis plus de 10 ans, il est grandement utilisé dans l’industrie vidéoludique et filmographique
	
	\subsubsection{Blender} 
		Blender, son alternative libre. Soutenue par la fondation blender, qui a par ailleurs réalisé quelques films d’animations tels que Big Buck Bunny ou Sintel, ses fonctionnalitées ne sont plus à prouver.
	
	\subsubsection{Unity} 
		Unity est un environnement et un moteur de rendu 3D ; initialement pensé pour les jeux vidéos, il permet de créer ou modifier des environnements en 3D et correspond parfaitement à nos besoins pour la modification du modèle d’appartement tremplin.
		De plus, contrairement à d’autres moteurs 3D, Unity propose une version gratuite, qui regroupe la majorité des fonctionnalités de la version payante à l'exception de la compilation 64b et ???.
		Il permet de réaliser toutes les actions classiques d'un logiciel de ce type, comme construire des objets, les animer, interagir avec, etc. Toutes ces actions sont effectuées de manière native depuis l'interface en écrivant des scripts C# ou Javascript via l'API fournie.
		Le développement y est grandement facilité, car celle-ci inclus de nombreuses fonctionnalités.
		Il n'y a donc pas besoin de sortir du logiciel pour développer.
		Unity est actuellement utilisé dans la salle Immersia de l’Irisa.      

	\subsubsection{VRPN}
		VRPN (Virtual-Reality Peripheral Network) est un système de gestion de périphériques pour la réalité virtuelle. Cette bibliothèque offre une interface entre le matériel et l'application. Elle offre des classes génériques pour chaque type de périphériques; par exemple, tout les trackers sont gérés de la même façon.
		Le projet fut initié en 1998 par Russell M. Taylor II de l'université de Caroline du Nord et est aujourd'hui maintenu par une vingtaine de contributeurs.
	
	\subsubsection{MiddleVR}
		MiddleVR est un plugin compatible avec Unity qui s'appuit sur  VRPN. Développé par I’m in VR (PARIS, France) permettant de gérer les interactions entre l’utilisateur et son environnement, et spécialement conçu pour les environnements en réalité virtuelle. 
		L'objectif de MiddleVR est de pouvoir programmer l'application en s'abstrayant des spécificités matérielles pour qu'elle soit déployable partout.
		Il propose une couche d’abstraction entre les périphériques et Unity. Ces périphériques comprennent ceux d’entrée (de capture), tels que les classiques claviers et souris mais aussi les bras à retour de force ou des MS Kinect, ainsi que ceux de sortie (de restitution), comme les écrans, les vidéoprojecteurs 3D mais aussi le son ou le retour de force des bras. 
		MiddleVR gère nativement la stéréoscopie active ou passive,
		- Interaction devices like 3D trackers (see full list on the right),
		- Clustering: Scenelock, swaplock both sofware \& hardware,
		- 3D interactions: Navigation, manipulation, menus and custom graphical user interfaces
		MiddleVR = s'abstraire des interfaces homme-machine - 

	\subsubsection{\#Five}
		\#Five est une bibliothèque pour Unity developpée en interne au sein de l’Irisa. \#Five ajoute plusieurs couches, parmi lesquelles : 
		Donner la mécanique qui fait les relations entre les objets pour gérer réactivité
		Infrastructure gérant le travail collaboratif sur environnement virtuel, c'est à dire quand plusieurs personnes intéragissent sur la même scène.
		Scénario de haut niveau, si je veux démonter une culasse, il faut démonter les boulons, pour démonter les boulons : Si je fais le 1 alors ensuite il faut faire le X, mais si je fais le 2 en premier alors il faut faire le Y ensuite
		Objets réactifs (décor, objets, inter relations) + scénario + humains virtuels pouvant collaborer


\subsection{Matériels et environnement technique}

Ce projet consistant en l'utilisation d'un environnement 3D virtuel (appartement de Kerpape aidant à la réhabilitation de personnes lourdement handicapées), où l'utilisateur sera amené à avoir des interactions avec cet environnement, nous allons donc avoir accès à la salle de réalité virtuelle $\mu$RV de l'INSA Rennes et à son matériel. à nous d'utiliser ce dernier à bon escient, pour répondre au mieux à la demande de Kerpape et pour pouvoir proposer un environnement d'apprentissage le plus performant possible notamment au niveau des interactions avec l'utilisateur. 
Ci-dessous vous trouverez la présentation du matériel présent dans la salle $\mu$RV et l'intérêt que pourrait présenter leur utilisation dans le projet.

\subsubsection{Matériel d'immersion}
Nous disposons de différents outils pour immerger l'utilisateur au coeur de l'environnement virtuel : 
\\

\textbf{Lunettes nVidia 3D Vision et Récepteur 3D Vision}
\\

Ces lunettes, sur batterie, permettent de visualiser une stéréoscopie active. Elles sont reconnues par l'ordinateur grâce à une base quiémet des signaux infrarouges.
\\

\textbf{Lunettes 3D Vuzik Wrap 920}
\\

Les lunettes 3D Vuzix Wrap 920 disposent de deux écrans intégrés ainsi que d'un capteur de mouvements.
\\

\textbf{Vidéoprojecteur 3D}
\\

Le vidéoprojecteur permet d'avoir un écran 3D à disposition pour s'immerger dans l'environnement plus facilement avec une qualité graphique bien supérieure à celle d'un écran d'ordinateur classique.
\\


\textbf{Oculus Rift}
\\

Le dispositif se démarque des systèmes comparables expérimentés précédemment par la très courte latence dans le suivi des mouvements de la tête et par l'important champ de vision offert. L'appareil se présente sous la forme d'un masque recouvrant les yeux et attaché au visage par une sangle ferméeà l'arrière du crâne. Un écran plat numérique est placéà quelques centimètres en face des yeux, perpendiculairementà l'axe du regard. Cet écran affiche une image stéréoscopique déformée numériquement pour inverser la distorsion optique créée par deux lentilles situées en face de chaque œil. Divers capteurs permettent de détecter les mouvements de tête de l'utilisateur, ce qui permet d'adapter en temps réel l'image projetée sur l'écran, afin de produire l'illusion d'une immersion dans la scène restituée.
\\

\textbf{Plateforme Immersia}
\\

En forme de « L », la salle Immersia est dotée d'un équipement immersif plongeant l?utilisateur dans un monde visuel et auditif de haute qualité. 
Elle est constituée  :
\begin{itemize}
  \item d'un système visuel utilisant 11 projecteurs : 8 Barco Galaxy NW12 et 3 Barco Galaxy 7+.
  \item d'un écran de verre de 9,60 mètres de long où sont projetées par l?arrière, les images stéréoscopiques.
  \item d'un système de localisation ART permettant à des objets réels d?être localisés à l?intérieur de la plate-forme.
  \item d'un système de rendu sonore fourni par un processeur Yamaha, lié soit à des hauts-parleurs Genelec au format sonore 10.2, soit à des casques Beyer Dynamic avec un format sonore virtuel de 5.1, contrôlé par la position de l?utilisateur.
\end{itemize}

\subsubsection{Matériel d'interaction}
Une fois l'utilisateur intégré dans la modélisation virtuelle de l'appartement, nous avons à notre disposition plusieurs équipements pour le faire interagir avec son environnement :
\\

\textbf{Microsoft Kinect}
\\

La Kinect développée par Microsoft, est le périphérique le plus intéressantà exploiter dans le cadre de la réalité virtuelle. Elle permet en effet de détecter la présence de 6 personnes et de suivre les mouvements de deux utilisateurs actifs grâceà ses lentilles placées sur un socle motorisé. Sa portée est comprise entre 1,2m et 3,5m. L'intérêt principal de la Kinect est que l'utilisateur puisse se déplacer librement dans une pièce sans avoirà manipuler une quelconque manette.
\\

\textbf{WiiMote, Nunchuck et WiiMotionPlusInside}
\\

La Wiimote est la manette fournie avec la console de jeu Wii, développée par Nintendo. Elle est composée de deux parties : la manette principale et le Nunchuk. Bien que moins pratique que la Kinect (l'utilisateur est contraint de manipuler une manette), la Wiimote dispose d'un système de liaison bluetooth d'une portée de 10m permettant une utilisation sans fil. L'avantage de la Wiimote réside dans le nombre et la précision des informations qu'elle est capable de fournir au programme. 
\newline
Elle permet en effet de :
\begin{itemize}
  \item mesurer les accélérations selon 3 axes (axes naturels 3D) grâces aux accéléromètres placés dans la manette principale et le Nunchuk
  \item mesurer l'inclinaison de la manette principale selon les 3 axes naturels
  \item mesurer la distance ainsi que la position entre la Wiimote et la barre infrarouge (référentiel).
\end{itemize}

\textbf{Joystick Extreme 3D Pro Logitech}
\\

Avec ses commandes avancées et sa gouverneà manche rotatif, ce joystickà retour de force stable et précis est prévu pour être connecté à un ordinateur et est utilisé pour des jeux de combat aérien acrobatique. Il est composé d'une gouverneà manche rotatif et de nombreux boutons.
\\

\textbf{Bras à retour de force Novint Falcon}
\\

Le Falcon, de la société Novint, est un périphérique haptique branché en USB. Il permet de ressentir le retour d'effort, et donc la texture et la résistance des objets, leur poids... Une boule est reliée au support par des tiges. C'est elle que l'on déplace et qui transmet à la main de l'utilisateur les efforts des moteurs.
\\

\subsection{Techniques d'interaction}
Au vu de toutes les ressources technologiques dont nous disposons, nous avons décidé de privilégier les techniques d'interaction suivantes :

\subsubsection{Interface Clavier/Souris et écran d'ordinateur}
Une première version utilisable sur un ordinateur avec ses périphériques de base (clavier et souris) permettant d'avoir un environnement d'apprentissage fonctionnel et testable rapidement, et également très portable.

\subsubsection{Compatibilité avec tous les périphériques via MiddleVR}
L'objectif serait de présenter une application fonctionnant avec tous les périphériques à notre disposition et cités précédemment, grâce à une reconnaissance et une configuration automatique des périphériques.
\\
