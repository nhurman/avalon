\section{Contexte}
%\subsection{\'Etude de l'existant}


\begin{comment}Introduction plus générale, dans le monde de l’informatique, pas que le projet, définitions associées (RV, interactions, immersion, domotique)
Qqes éléments sur l’impact sociétal de l’aide à la personne 
Présentation sujet (fait le lien), partenariat INSA/Kerpape/(3è partenaire) ; mettre en avant le fait que demandeur extérieur \end{comment}

	La notion de réalité virtuelle, contrairement à ce que l'on pourrait penser, n'est pas récente, et date du début des années 80 quand Jaron Lanier, un informaticien américain pionnier du domaine, l'a popularisée. Cette notion n'a pourtant pas, à l'origine, l'exact sens qu'on lui prête habituellement aujourd'hui : une réalité virtuelle sous-entendrait qu'il s'agit d'une copie exacte de la réalité, ce qui n'est jamais le cas faute de moyens techniques, et n'est pas toujours recherché. Le terme venant  de l'anglais, \emph{virtual} peut se traduire par \emph{virtuelle} mais aussi par \emph{quasi} ou \emph{pratiquement}, or cette notion de \emph{quasi-réalité} correspondrait mieux à ce qu'est effectivement la réalité virtuelle. En effet, on cherche souvent à modifier la réalité %parler de visions abstraites de la réalité, comme notre truc avec les interrupteurs, vue symbolique !
\\

	La réalité virtuelle est un domaine scientifique qui a pour finalité « de permettre à une personne [...] une activité sensori-motrice et cognitive dans un monde [...] créé numériquement ». 
%Il faudra penser à citer la source : Le traité de la RV
L'objectif n'est donc pas, une fois encore, de créer un substitut exact à la réalité, mais plutôt une version plus ou moins modifiée qui correspondent à nos besoins. 
\\

	La notion de réalité virtuelle implique donc également celles d'immersion et d'interaction, elles sont en fait constitutives de ce  qu'est la réalité virtuelle. L'utilisateur doit pouvoir faire des actions motrices sur son environnement, il doit pouvoir agir dessus que ce soit par les mouvements, la parole, les gestes... Il n'y a pas de règle définie tant qu'il s'agit d'une activité motrice. 
	Une activité sensorielle, par ailleurs, signifie que l'utilisateur percevra un impact de ses actions sur le monde virtuel. Encore une fois il n'y a pas de liste de réponses sensorielles acceptables, ce peut être un son, une modification de l'affichage...	%Eventuellement : le graphe p. 9 du Traité