\section{Contexte}


%Introduction plus générale, dans le monde de l’informatique, pas que le projet, définitions associées (RV, interactions, immersion, domotique)
%Qqes éléments sur l’impact sociétal de l’aide à la personne 
%Présentation sujet (fait le lien), partenariat INSA/Kerpape/(3è partenaire) ; mettre en avant le fait que demandeur extérieur 
	La notion de réalité virtuelle, contrairement à ce que l'on pourrait penser, n'est pas récente, et date du début des années 80 quand Jaron Lanier, un informaticien américain pionnier du domaine, l'a popularisée. Cette notion n'a pourtant pas, à l'origine, l'exact sens qu'on lui prête habituellement aujourd'hui : une réalité virtuelle sous-entendrait qu'il s'agit d'une copie exacte de la réalité, ce qui n'est jamais le cas faute de moyens techniques, et n'est pas toujours recherché. Le terme venant  de l'anglais, \emph{virtual} peut se traduire par \emph{virtuelle} mais aussi par \emph{quasi} ou \emph{pratiquement}, or cette notion de \emph{quasi-réalité} correspondrait mieux à ce qu'est effectivement la réalité virtuelle. 
\\

En effet, de manière plus formelle, on peut définir la réalité virtuelle comme suit :\begin{quote}
%« 
La réalité virtuelle est un domaine scientifique et technique exploitant l'informatique et les interfaces comportementales en vue de simuler dans un monde virtuel le comportement d'entités 3D, qui sont en interaction en temps réel entre elles et avec un ou des utilisateurs en immersion pseudo-naturelles par l'intermédiaire de canaux sensori-moteurs.
% »
\end{quote}
%Il faudra penser à citer la source : Le traité de la RV
Cela signifie que pour que l'on puisse parler de réalité virtuelle il faut que plusieurs conditions soient réunies :	
\begin{itemize}
\item Les interfaces comportementales désignent les interfaces entre l'utilisateur et le monde virtuel. Il existe dans un premier temps des interfaces motrices, qui permettent de reconnaître les différentes actions que l'utilisateur peut entreprendre (mouvements, voix, etc) pour que le système informatique gérant le monde virtuel puisse les prendre en compte. Les interfaces sensorielles ensuite font la liaison dans l'autre direction et informent l'utilisateur de l'état du monde virtuel et des modifications éventuelles que ses actions ont entraîné (images, sons, etc). Ces deux types d'interface permettent de gérer l'interaction avec le monde virtuel et sont donc nécessaires pour parler de réalité virtuelle. 
\item L'immersion caractérise le fait que les interfaces entre l'utilisateur et le monde virtuel se fassent oublier, et ce, en ressemblant le plus possible aux méthodes d'interaction que nous avons avec le monde réel. Elle ne peut pas être parfaite, car il y a toujours du matériel à utiliser qui n'est pas nécessaire pour interagir avec le monde réel (lunettes 3D, WiiMote, etc) mais doit être la plus totale possible.
\end{itemize}


La notion de réalité virtuelle implique donc également celles d'immersion et d'interaction, elles sont en fait constitutives de ce  qu'est la réalité virtuelle. L'utilisateur doit pouvoir faire des actions motrices sur son environnement, il doit pouvoir agir dessus que ce soit par les mouvements, la parole, les gestes... Il n'y a pas de règle définie tant qu'il s'agit d'une activité motrice. 
Une activité sensorielle, par ailleurs, signifie que l'utilisateur percevra un impact de ses actions sur le monde virtuel. Encore une fois il n'y a pas de liste de réponses sensorielles acceptables, ce peut être un son, une modification de l'affichage...	%Eventuellement : le graphe p. 9 du Traité