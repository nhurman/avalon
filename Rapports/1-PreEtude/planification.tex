\section{Planification}

L'un des buts de ce rapport est de prevoir l'organisation de notre travail ainsi que de faire une estimation du planning de nos productions.

\subsection{Versions intermédiaires}
Lors de la réalisation de ce projet, nous allons produire plusieurs versions intermédiaires pour nous permettre de construire chaque fonctionnalité au fur et à mesure :
\begin{itemize}
  \item Version \textnumero1 : Implémentation de l'appartement complet dans un environement 3d. Le personnage peut se déplacer librement à l'intérieur, mais sans collisions avec l'environnement.
  \item Version \textnumero2 : Une deuxième version ajoute les interactions basiques : collision avec les obstacles, possibilité d'utiliser les interrupteurs. Elle correspond au mode \enquote{Utilisation} du logiciel.
  \item Version \textnumero3 : La troisième version permet d'utiliser le mode \enquote{Apprentissage}.
  \item Version \textnumero4 : La quatrième version implémente les différents scénarios d'utilisation du logiciel et fonctionne avec PC+clavier/souris.
  \item Version \textnumero5 : La version finale fonctionne en réalité virtuelle dans ces 3 envionnements : salle $\mu$RV, salle Immersia, casque de réalité virtuelle.
\end{itemize}

\subsection{Organisation du travail}
Pour ce projet, nous avons mis en place un certain nombre d'outils de travail collaboratif :

\begin{itemize}
  \item GitHub : pour nos documents versionnés tel que le code source ;
  \item Cloud privé : pour les documents confidentiels ;
  \item Google Drive : pour tout autre document ;
  \item Wiki : permet une base centrée de connaissance sur la salle $\mu$RV et de tous les projets qui y ont été réalisés.
\end{itemize}


\subsection{Estimation du planning}
Au cours de l'année, nous sommes chargés de rédiger 6 documents (rapports, documentation) et de réaliser un certain nombre de versions intermédiaires de notre logiciel.
Voici une estimation du planning pour l'année :
\\
\\

\begin{tabular}{|l|l|}
\hline
  Date &
  Production \\
\hline
  19 novembre &
  Rapport de spécification fonctionnelle VP \\
\hline
  27 novembre &
  Rapport de spécification fonctionnelle VF \\
\hline
  9 décembre &
  Version PC \textnumero1 \\
\hline
  9 décembre &
  Dossier de planification Initial VP \\
\hline
  17 décembre &
  Dossier de planification Initial VF \\
\hline
  9 janvier &
  Version PC \textnumero2 \\
\hline
  6 février &
  Rapport de conception logicielle VP \\
\hline
  9 février &
  Version PC \textnumero3 \\
\hline
  12 février &
  Rapport de conception logicielle VF \\
\hline
  9 mars &
  Version PC \textnumero4 \\
\hline
  25 mars &
  Page HTML VP \\
\hline
  2 avril &
  Page HTML VF \\
\hline
  9 avril &
  Version PC \textnumero5 \\
\hline
  16 mai &
  Documentation en Ligne VP \\
\hline
  21 mai &
  Rapport Final/Annexes + Bilan Planification \\
\hline
  26 mai &
  Rapport Final/Annexes + Bilan Planification \\
\hline
  28 mai &
  Documentation en Ligne VF \\
\hline
\end{tabular}

