\section{Planification}
\subsection{Versions intermédiaires}
Lors de la réalisation de ce projet, nous allons produire plusieurs versions intermédiaires pour nous permettre de construire chaque fonctionnalité au fur et à mesure :
\begin{itemize}
  \item Version \textnumero1 : On commence par une version qui implémente l'appartement complet. Le personnage paut se déplacer librement à l'intérieur, mais sans collisions avec l'environnement.
  \item Version \textnumero2 : Une deuxième version ajoute les interactions basiques : collision avec les obstacles, possibilité d'utiliser les interrupteurs. Elle correspond au mode "Utilisation" du logiciel.
  \item Version \textnumero3 : La troisième version permet d'utiliser le mode "Apprentissage".
  \item Version \textnumero4 : La version finale implémente les différents scénarios d'utilisation du logiciel.
\end{itemize}

\subsection{Organisation du travail}
Pour ce projet, nous avons mis en place un certain nombre d'outils de travail collaboratif :

\begin{itemize}
  \item GitHub : pour nos documents versionnés tel que le code source
  \item Cloud privé : pour les documents confidentiels
  \item Google Drive : pour tout autre document
  \item Wiki : permet une base centrée de connaissance sur la salle $\mu$RV et tout les projets qui y ont été réalisés
\end{itemize}


\subsection{Estimation du planning}
Au cours de l'année, nous sommes chargés de rédiger 6 papiers (rapports, documentation) et de réaliser un certain nombre de versions intermédiaires de notre logiciel.
Voici une estimation du planning pour l'année :

\begin{tabular}{|l|l|}
\hline
  Date &
  Production \\
\hline
  19 Octobre &
  Rapport de spécification fonctionnelle VP \\
\hline
  27 Octobre &
  Rapport de spécification fonctionnelle VF \\
\hline
  9 Décembre &
  Version PC \textnumero1 \\
\hline
  9 Décembre &
  Dossier de planification Initial VP \\
\hline
  16 Décembre &
  Version PC \textnumero2 \\
\hline
  17 Décembre &
  Dossier de planification Initial VF \\
\hline
  6 Février &
  Rapport de conception logicielle VP \\
\hline
  12 Février &
  Rapport de conception logicielle VF \\
\hline
  25 Mars &
  Page HTML VP \\
\hline
  2 Avril &
  Page HTML VF \\
\hline
  16 Mai &
  Documentation en Ligne VP \\
\hline
  21 Mai &
  Rapport Final/Annexes + Bilan Planification \\
\hline
  26 Mai &
  Rapport Final/Annexes + Bilan Planification \\
\hline
  28 Mai &
  Documentation en Ligne VF \\
\hline
\end{tabular}

