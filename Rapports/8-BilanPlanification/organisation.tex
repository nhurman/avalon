\section{Organisation collective}

\subsection{Outils collaboratifs}

Voici les quelques outils principaux qui nous ont permis de travailler efficacement en groupe, en confiant notamment la gestion de version et les fusions entre celles-ci.

\subsubsection{owncloud}

Pour partager avec les membres du groupe les fichiers volumineux et qui évoluent peu (ou pas), nous avons utilisé OwnCloud.
Cette solution nous a permis de partager les fichiers que Kerpape nous avait fourni par exemple. 
C'est un logiciel de stockage en ligne, hébergé sur un serveur d'un des membres du groupe, pour des questions de non-partage avec l'extérieur, les fichiers de Kerpape étant leur propriété.
Cette solution avait de nombreux avantages, parmi lesquelles sa simplicité d'utilisation : une interface web pour voir les documents, les télécharger ou les téléverser.

\subsubsection{OneDrive}

OneDrive est le service en ligne de la suite Microsoft Office.
Il a permis notamment de préparer les diaporamas pour les soutenances et pour les nombreuses présentations du projet que nous avons faites.
Grâce à OneDrive en effet, tous les membres du groupe ont pu travailler sur différentes parties d'une même présentation, mais dans un style unifié, sans problème de fusions desdites parties.

\subsubsection{Git et Github}

Outil pour le moins essentiel dans le fonctionnement en groupe, le gestionnaire de version Git a été d'une grande aide.
Capable de traiter automatiquement les différentes versions des fichiers, il aura néanmoins nécessité de passer le fichier Unity principal en mode texte, ceci en vue de simplifer les fusions de versions.

Nous avons également utilisé un dépôt privé github pour héberger les multiples fichiers du projet : rapports, objets 3d, scripts, textures ....
Celui-ci a l'avantage d'être tout le temps disponible, et de n'être accessible qu'aux membres du projet.

\subsubsection{\LaTeX}

C'est en \LaTeX que nous avons rédigé tous nos rapports.
En effet, ce langage permet de faire des documents soignés, rigoureux sans avoir à perdre trop de temps sur la mise en page.
Mieux encore, \LaTeX n'étant constitué que de simples fichiers textes, la gestion des versions de rapports a été possible sous git, facilitant grandement l'interopérabilité entre les membres sur les phases de rédaction de rapports.

\subsection{Ressenti global des membres de l'équipe}

Dans l'ensemble, nous sommes satisfaits de notre organisation de groupe. 
Nous avons su mettre en place une ambiance de travail conviviale mais sérieuse. 
Ainsi nous n'avons pas souffert de gros désaccords au sein de l'équipe, facilitant l'avance générale.

Concernant notre mode de gestion de projet, nous pensons que désigner un chef de projet par livrable a permis à chacun de prendre des responsabilités, sans faire reposer le poids d'un projet entier sur les épaules d'une seule personne.
De plus pour la version finale de la solution logicielle, étant tous garants de son bon avancement, nous nous sommes mobilisés y compris dans des périodes moins propices à avancer sur le projet.

Concernant les estimations et l'avancement tout au long du projet le bilan est plus mitigé.
Même si nous avons respecté les jalons externes, les livraisons auprès de Kerpape n'ont pas été aussi fréquentes que prévues.
 
Nos estimations étaient un peu faibles à l'égard de certains aspects du projet ; ce problème est lié au fait que dans le groupe, parmi les personnes présentes au second semestre, personne n'avait travaillé ni sur Unity, ni sur MiddleVR.

Dans l'ensemble, nous sommes très contents d'avoir pu travailler sur un projet comme celui-ci, et de nous confronter à des contraintes de travail liées à des groupes plus grands et un temps de travail réparti sur une année.
Le projet nous a également permi de nous former à faire de la documentation, du reporting et de saisir leur importance dans un tel contexte. 