\section{Architecture logicielle}
Pour ce projet, plusieurs logiciels nous ont été imposé. Cette partie détaille comment ces logiciels seront utilisés pour concevoir l'application.

\subsection{Unity}
Unity est un moteur de jeu et environnement de développement. Bien que Unity donne la possibilité d'insérer des scripts pour gérer les objets plus en profondeur, il permet avant tout de créer une scà¨ne 3d. L'utilisateur peut se deplacer à  l'interieur, en ayant des interactions avec l'environnement. Unity gà¨re un certain nombre d'éléments, dont :
\begin{itemize}
        \item les colliders : représente les collisions engendrées par l'objet.
        \item les meshs : correspond à l'affichage des objets de la scène.
\end{itemize}


\subsection{MiddleVR}
MiddleVR est un plugin Unity utilisé pour s'abstraire des périphériques d'entrée/sortie. Il permet de créer un fichier de configuration contenant les informations sur les différents périphériques dont se servira l'application. Pour cela, MiddleVR possède sa propre interface et il n'est pas nécessaire d'écrire du code.
Pour récupérer les valeurs des périphériques dans Unity, on utilise le VRManager, qui sert d'interface entre Unity et MiddleVR.

\subsection{Script C\#}
Unity utilise des scripts en C\# pour réaliser des interactions plus poussées que de simples collisions. Ces scripts nous donne la possibilité d'allumer ou éteindre une lumià¨re, ou déplacer un objet par exemple. Ils sont l'unique source de code que nous allons produire dans ce projet, il devront donc gérer les différents modes d'apprentissage, ainsi que les scénarios d'utilisation.
Dans Unity, un script est attaché à  un objet, il correspond à  son comportement. Par exemple, un script attaché à  un interrupteur va indiquer ce qu'il se passe lorsque le joueur appui sur cet interrupteur. Il va commander le changement d'état de l'interrupteur, le changement de lumià¨re (si il commande la lumià¨re d'une pià¨ce) et éventuellement d'autres éléments selon le mode d'apprentissage.
