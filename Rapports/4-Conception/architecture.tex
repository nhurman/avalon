\section{Architecture logicielle}
Pour ce projet, plusieurs logiciels nous ont été imposés. Cette partie détaille comment ces logiciels seront utilisés pour concevoir l'application.

\subsection{Unity}
\begin{figure}[h]
  \includegraphics[width=1\textwidth]{4-Conception/img/unity_screenshot.png}
  \caption{Capture d'écran : Unity}
  \label{unity}
\end{figure}

Unity est un moteur de jeu et plus généralement un environnement de développement. Bien qu'Unity donne la possibilité d'insérer des scripts pour gérer les objets plus en profondeur, il permet avant tout de créer une scène 3D de manière intuitive à travers son interface graphique. L'utilisateur peut se deplacer à l'interieur et les diverses entités de la scène réagir à des évènements. Unity gère un certain nombre d'éléments, dont :
\begin{itemize}
        \item les colliders, qui représentent les collisions engendrées par l'objet ;
        \item les meshs, qui correspondent à l'affichage des objets de la scène.
\end{itemize}


\subsection{MiddleVR}
\begin{figure}[h]
  \includegraphics[width=1\textwidth]{4-Conception/img/middlevr_screenshot.png}
  \caption{Capture d'écran : MiddleVR}
  \label{middlevr}
\end{figure}
MiddleVR est un plugin Unity utilisé pour s'abstraire des périphériques d'entrée/sortie. Il permet de créer un fichier de configuration contenant les informations sur les différents périphériques dont se servira l'application. Pour cela, MiddleVR possède sa propre interface et il n'est pas nécessaire d'écrire du code.
Pour récupérer les valeurs des périphériques dans Unity, on utilise le VRManager, qui sert d'interface entre Unity et MiddleVR.

\subsection{Scripts}
Unity utilise des scripts, qui peuvent être écrits en Javascript, C\# ou Boo, pour réaliser des interactions plus poussées que de simples collisions. Ces scripts nous donnent la possibilité d'allumer ou éteindre une lumière, ou déplacer un objet par exemple. Ils sont l'unique type de code que nous allons produire dans ce projet, il devront donc gérer les différents modes d'apprentissage, ainsi que les scénarios d'utilisation.
Dans Unity, un script est attaché à un objet, il correspond à son comportement. Par exemple, un script attaché à un interrupteur va indiquer ce qu'il se passe lorsque le joueur appuie dessus. Il va commander le changement d'état de l'interrupteur, le changement de lumière (s'il commande la lumière d'une pièce) et éventuellement d'autres éléments selon le mode d'apprentissage.
\begin{figure}[h]
\centering
\includegraphics[width=1\textwidth]{4-Conception/img/recap.png}
\caption{ Diagramme récapitulatif }
\label{recap}
\end{figure}

Dans la figure \ref{recap}, on peut noter que nous ne communiquons pas avec le VRPN nous-même, MiddleVR s'en charge à notre place. Plus d'informations sur le VRPN peuvent être trouvées dans la documentation officielle \footnote{\url{http://www.middlevr.com/middlevr-for-unity/}} ou dans notre rapport de pré-étude.
