\section{Diagrammes comportementaux}

\subsection{Lancement de l'application}
\begin{figure}[h]
\centering
\includegraphics[width=1\textwidth]{4-Conception/img/diagActivite.png}
\caption{ Diagramme d'activité}
\end{figure}

Au lancement du logiciel, l'utilisateur peut personnaliser la simulation à travers plusieurs options.

Tout d'abord, il peut choisir un mode d'apprentissage. Ces modes apportent de l'autonomie ou donnent des directions à l'utilisateur en lui donnant des indicateurs visuels sur ce qu'il peut faire. Voici les différents modes :
\begin{itemize}
\item Symbolique ;
\item Assisté ;
\item Autonome.
\end{itemize}
\vspace{0.5cm}

De plus, il est possible de jouer un scénario, pour forcer l'utilisateur à faire certains choix et interagir avec certains objets. Voici les différents scénarios :
\begin{itemize}
\item Appel téléphonique normal ;
\item Appel \textit{via} l'interphone d'une personne venant fréquemment (un infirmier par exemple) ;
\item Appel \textit{via} l'interphone concernant une visite inattendue.
\end{itemize}
\vspace{0.5cm}

Ces différents choix seront gérés par des scripts C\# de Unity, et plus précisement par le GameManager, décrit plus loin dans ce rapport.

\subsection{Diagramme des cas d’utilisations}
\begin{figure}[h]
    \centering
    \includegraphics[width=\textwidth]{4-Conception/img/diagCasUsage.png}
    \caption{Diagramme de cas d'utilisation : Menus}
\end{figure}

\subsection{Diagramme de séquence}
\begin{figure}[h]
    \centering
    \includegraphics[width=\textwidth]{4-Conception/img/diagSequenceLight.png}
    \caption{Diagramme de séquence : déroulement d'une interraction}
    \label{fig:sequence_diagram}
\end{figure}
