\section{Diagramme de classes}

Sur le diagramme, une classe UI_Menu est présente; la dernière version de MiddleVR rajoute des classes d'interfaces adaptées à la 3D.
Cette version étant disponible seulement depuis la semaine précédente, nous n'avons pas eu l'occasion d'étudier l'architecture proposée pour cette partie.

Bon nombre de classes comme \textit{VariableLight}, \textit{Shutter} ou \textit{Table} ne disposent pas de variable numérique modélisant leur état courant.
Il faut prendre en compte que ce sont des scripts associé un objet 3D. La valeur actuelle (de hauteur dans le cas de la table par exemple) correspond à une ou plusieurs caractéristiques graphiques de l'objet.
Le script inclu une valeur minimale et maximale pour la valeur contrainte, ainsi qu'un pas de déplacement.
Ce pas permet de régler le changement de valeur à chaque clic; par exemple, on choisira 1cm dans le cas de la table, pour que le déplacement soit cohérent avec la réalité.
 
 
 