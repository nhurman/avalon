\section{Conclusion}

Cette phase de conception nous a permis d'avoir une idée définitive et arrêtée des différents aspects de notre application pour ceux qui étaient encore sujets à discussion, et d'entériner le tout au moyen des différents diagrammes de ce rapport. Ainsi, l'architecture logicielle, mais aussi la représentation des différents scénarios d'utilisation qui seront implémentés, et le déroulement d'une session d'utilisation d'Avalon, ont été détaillés.

Cette phase de conception a aussi été pour nous l'occasion de découvrir un paradigme de programmation qui, s'il approche de la programmation orientée objet, en différe tout de même et nous a fait appréhender la modélisation d'un point de vue différent : la programmation orientée composants de Unity.

Cette phase est la dernière qui précède la partie principalement dédiée à la réalisation de notre application. Nous l'avons donc menée avec minutie pour pouvoir pleinement nous appuyer dessus par la suite, ainsi que pour avoir un support d'échange clair avec le centre de Kerpape qui leur donnera une vision la plus précise possible de l'aspect final que prendra le projet. Nous espérons ainsi leur permettre de le suivre le plus précisément possible et d'intervenir, par exemple en cas d'incompréhension, pour que le résultat final soit à la hauteur de leurs attentes.