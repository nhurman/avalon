\section{Introduction}
Avalon est un projet proposé par le centre de rééducation de Kerpape pour permettre à des personnes handicapées de s'exercer à l'utilisation des équipements domotiques dans les appartements tremplin spécialement équipés du centre.

Après les phases de spécification et planification, nous avons sommes attelés à la phase de conception logicielle e notre projet. Nous allons présenter dans ce rapport la modélisation logicielle que nous avons établie grâce au standard UML. Comme nous l'expliquerons plus loin, une partie de la modélisation nous a été rendue compliquée par le fait que la programmation \textit{via} Unity3D ne correspond pas tout à fait à la programmation orientée objet que nous avons rencontrée jusque là : nous sommes dans un cas particulier, qui est de la programmation orientée composants. 

Une modélisation détaillée nous permettra de structurer la phase de développement de notre application et de travailler sur des bases saines, car chacun saura précisément à quoi l'on doit aboutir. Elle nous permet aussi de fixer nos idées sur les derniers détails qui restaient en débat. 

Nous allons dans un premier temps détailler l'architecture logicielle : les technologies et les logiciels que nous allons utiliser et la façon dont ils interagissent entre eux ; quel va être leur impact sur la modélisation finale du projet. 

Ensuite, nous présenterons la modélisation UML à proprement parler d'Avalon. Les diagrammes de cas d'utilisation qui représentent les trois différents scénarios possibles que nous allons implémenter, les diagrammes de classe et le diagramme d'activité qui résumera le déroulement d'une session.

Enfin dans une dernière partie il sera question de l'interaction entre les différentes briques logicielles d'un point de vue technique : l'intégration de C\# dans Unity et les contraintes qui en découlent, l'interfaçage de MiddleVR avec Unity, etc. 