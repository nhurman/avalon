\documentclass[a4paper,11pt]{article}
\usepackage{exptech}
\usepackage{textcomp}
\usepackage{graphicx}
\usepackage{array}
\usepackage[babel=true]{csquotes}
\usepackage{url}
\usepackage{hyperref}
\usepackage{wrapfig}
\usepackage[export]{adjustbox}
\usepackage{titletoc}

\hypersetup{
  %bookmarks=true, % show bookmarks bar?
  pdftitle={Avalon - Rapport de pré-étude}, % title
  pdfnewwindow=true, % links in new window
  colorlinks=true, % false: boxed links; true: colored links
  linkcolor=black, % color of internal links (change box color with linkbordercolor)
  citecolor=cyan, % color of links to bibliography
  filecolor=cyan, % color of file links
  urlcolor=cyan % color of external links
}

\title{
  \textbf{Avalon}\\
  Rapport final
}
\markright{Avalon - Rapport final}
\author{
\begin{minipage}{0.4\textwidth}
	\begin{flushleft} \large
		\emph{Auteurs :}\\
		Alexandre \textsc{Audinot}\\
		Julien \textsc{Bouvet}\\
		Thierry \textsc{Gaugry}\\
		Nicolas \textsc{Hurman}\\
		Alexandre \textsc{Leonardi}\\
	\end{flushleft}
\end{minipage}
\begin{minipage}{0.4\textwidth}
	\begin{flushright} \large
		\emph{Encadrants :} \\
		Valérie \textsc{Gouranton}\\
		Ronan \textsc{Gaugne}\\
		Bruno \textsc{Arnaldi}\\
		Willy \textsc{Allègre}\\
		Jean-Paul  \textsc{Departe}\\
	\end{flushright}
\end{minipage}
}

\date{26 mai 2015}

\begin{document}
\maketitle
\thispagestyle{empty}
\begin{abstract}
\textbf{Avalon :} Environnement de Réalité Virtuelle pour l'apprentissage à l'utilisation d'appartements tremplin. Réalisation en 3D d'un appartement domotisé interactif utilisé dans le cadre de la rééducation des personnes handicapées.\newline

Le projet est proposé par le centre mutualiste de rééducation et de réadaptation fonctionnelles de Kerpape (plus particulièrement les ingénieurs du laboratoire électronique Willy Allègre et Jean-Paul Departe).\newline

Le modèle 3D de l'appartement nous est fourni, et notre travail consiste à réaliser un logiciel fonctionnel permettant de se déplacer dans l'appartement et implémentant les interactions avec les différents éléments de domotique, en plus de prendre en charge différents périphériques de contrôle. 
\end{abstract}

\begin{figure}[h!]
	\centering
	\includegraphics[width=0.7\textwidth]{8-BilanPlanification/img/screen_appart.png}
\end{figure}

%\vfill
%[width=\textwidth]
\begin{figure}[h!]
   \begin{minipage}{0.3\linewidth}
      \includegraphics[scale=0.9]{8-BilanPlanification/img/logo_insa.jpeg}
   \end{minipage} 
   \begin{minipage}{0.2\linewidth}
      \centering
      \includegraphics[scale=0.5,left]{8-BilanPlanification/img/logo_irisa.jpg}
   \end{minipage}\hfill
   \begin{minipage}{0.2\linewidth}
      \includegraphics[scale=0.9]{8-BilanPlanification/img/logo_kerpape.png}
   \end{minipage}
\end{figure}

\pagebreak

\tableofcontents
\pagebreak
\section{Planification Initiale}

\subsection{Jalons}

\subsubsection{Jalons fixées par l'INSA}

Ci-dessous les dates fixées par l'INSA, qui ont fait office de jalons pendant toute la durée du projet.\\

\begin{tabular}{|l|l|}
\hline
  Date &
  Production \\
\hline
  12 février &
  Rapport de conception logicielle  \\
\hline
  2 avril &
  Page HTML  \\
\hline
  26 mai &
  Rapport Final/Annexes + Bilan Planification \\
\hline
  28 mai &
  Documentation en Ligne VF \\
\hline
\end{tabular}


\subsubsection{Jalons internes}
Dans le but de gérer notre avancement dans le temps et d'obtenir des retours de la part de Kerpape, nous nous étions fixé des jalons \enquote{internes}.
Ceux-ci consistaient en un certain nombre de versions intermédiaires du logiciel à livrer, versions plus ou moins éloignées les unes des autres.
Nous avions fixé des niveaux de fonctionnalité pour chaque version, par exemple, la version n°3 devait intégrer les collisions dans la scène entre l'utilisateur et les divers objets.\\

\begin{tabular}{|l|l|}
\hline
  Date &
  Production \\
\hline
  9 janvier &
  Version PC \textnumero2 \\
\hline
  9 février &
  Version PC \textnumero3 \\
\hline
  9 mars &
  Version PC \textnumero4 \\
\hline
  9 avril &
  Version PC \textnumero5 \\
\hline
\end{tabular}\\

\subsection{Méthode de travail}

Les jalons imposés, les fonctionnalités prévues en avance et les seuils d'acceptation/validation que nous nous étions fixés nous ont naturellement conduits à travailler en suivant un cycle en V.
Les dates de rendus de rapports à l'INSA ont bien sûr fait office de jalons dans ce cycle en V.
En revanche, nos jalons internes (livrables pour Kerpape) ont ajouté à ce cycle en V une coloration agile, composée de retours et corrections.

Un autre aspect de cette coloration agile est du refactoring régulier.
En effet, que ce soit dans les scripts ou dans la hiérarchie du projet Unity, ou encore dans l'arbre de la scène, nous avions prévu que réorganiser les fichiers, la hiérarchie des objets ou le code nous prendrait un certain temps.

\subsection{Estimations}

Pour réaliser la solution pour Kerpape nous avions fait une estimation du nombre d'heures nécessaires.
Nos estimations étaient basées sur nos expériences personnelles, venant d'autres projets, et de divers retours d'expérience.
Nous avions estimé à environ 600 heures de travail les fonctionnalités restantes après le mois de février.
Dans ce planning nous avions compté les week-end, les séances de projets, les vacances mais pas les semaines de partiels ni les semaines de révisions.

\subsubsection{Chronologie du projet}
Cette vue (cf.\textsc{figure~\ref{fig:timeline}}) est la plus synthétique que nous ayons réalisée, elle donne une vue d'ensemble du projet ne comprenant que les plus grandes étapes. 
\begin{figure}[h]
	\centering
	\caption{Chronologie générale}
		\includegraphics[width=\textwidth]{8-BilanPlanification/img/timeline.PNG}
	\label{fig:timeline}
\end{figure}

\subsubsection{Temps de travail cumulé}
Le temps de travail cumulé (cf.\textsc{figure~\ref{fig:avancement}}) restant est assez équilibré tout au long du projet avec toutefois quelques variations brusques autour des jalons précédemment définis.
Ces variations sont liées au fait que nous avons dégagé plus de temps autour des jalons en cas de problèmes. 

\begin{figure}[h]
	\centering
	\caption{Temps de travail cumulé restant}
		\includegraphics[width=\textwidth]{8-BilanPlanification/img/avancement.PNG}
	\label{fig:avancement}
\end{figure}
\pagebreak
\section{Evolution tout au long du projet}
\pagebreak
\section{Méthodes d'organisation \& planification}
Au vu de l'envergure de notre projet, nous sommes obligés de prévoir une planification la plus détaillée possible, qui tienne compte de toutes les tâches qui compoeront la réalisation de notre application, accompagnées de la charge de travail que nous estimons devoir fournir pour les réaliser. Les différents livrables qui nous sont demandés feront office de jalons tout au long de notre projet. 

\subsection{Planification prévisionnelle}
Voici le planing, tel que nous l'avions estimé lors de notre 1\up{er} rapport : \newline
\begin{tabular}{|l|l|}
\hline
  Date &
  Production \\
\hline
  9 janvier &
  Version PC \textnumero2 \\
\hline
  6 février &
  Rapport de conception logicielle VP \\
\hline
  9 février &
  Version PC \textnumero3 \\
\hline
  12 février &
  Rapport de conception logicielle VF \\
\hline
  9 mars &
  Version PC \textnumero4 \\
\hline
  25 mars &
  Page HTML VP \\
\hline
  2 avril &
  Page HTML VF \\
\hline
  9 avril &
  Version PC \textnumero5 \\
\hline
  16 mai &
  Documentation en Ligne VP \\
\hline
  21 mai &
  Rapport Final/Annexes + Bilan Planification \\
\hline
  26 mai &
  Rapport Final/Annexes + Bilan Planification \\
\hline
  28 mai &
  Documentation en Ligne VF \\
\hline
\end{tabular}

\subsection{Méthode agile \& méthode en V}
Au cours du projet, 

Pour ce projet, nous devons produire plusieurs livrables : six rapports, une
documentation en ligne, une page HTML résumant notre projet ainsi qu’un ou plusieurs
exécutables, leur procédure d’installation, et enfin des jeux de tests.
Ces livrables sont dus à des dates dites « jalons », après les phases
correspondantes. Par exemple, le rapport de conception sera à rendre le 13 février,
après une phase de conception durant du 19 décembre au 13 février.
Ces dates « jalons » ainsi que les différentes phases sont définies de façon
commune pour tous les projets : nous ne pouvons pas passer outre et devons respecter
les délais imposés. C’est pourquoi nous avons préféré, après d’âpres discussions,
adopter une méthode de production en V (cf. Figure 3) plutôt qu’une méthode agile
avec des « sprints » qui nous semblait peu réalisable au vu de la durée probable de
ces sprints (un mois). De plus, des sprints d’un mois impliqueraient de faire de la
conception de façon permanente, alors que nous avons une phase de conception
prévue dans la définition du projet pour cela.

\subsection{Répartition des temps de travail}




\end{document}
