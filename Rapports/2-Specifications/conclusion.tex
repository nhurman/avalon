\section{Livrables}
Lors de la réalisation de ce projet, nous allons produire plusieurs versions intermédiaires pour nous permettre de construire chaque fonctionnalité au fur et à mesure :
\begin{tabular}{|l|c|p{8cm}|}
  \hline
  Date & Version & Fonctionnalités \\
  \hline
  9 décembre & Version PC 1 & Implémentation de l'appartement complet dans un environnement 3d. Le personnage peut se déplacer librement à l'intérieur, mais sans collisions avec l'environnement. \\
  \hline
  9 janvier & Version PC 2 & Une deuxième version ajoute les interactions basiques : collision avec les obstacles, possibilité d'utiliser les interrupteurs. Elle correspond au mode \enquote{Utilisation} du logiciel. \\
  \hline
  9 février & Version PC 3 & La troisième version permet d'utiliser le mode \enquote{Apprentissage}. \\
  \hline
  9 mars & Version PC 4 & La quatrième version implémente les différents scénarios d'utilisation du logiciel et fonctionne avec PC+clavier/souris. \\
  \hline
  9 avril & Version PC finale & La version finale fonctionne en réalité virtuelle dans ces 3 environnements : salle $\mu$RV, salle Immersia, casque de réalité virtuelle.\\
  \hline
\end{tabular}

\section{Conclusion}
Notre logiciel est destiné à un large éventail de personnes ; n'ayant pas forcément des compétences en informatique très développées.
\`A travers les diagrammes de cas d'utilisation et d'interaction, nous avons détaillé une ergonomie qui nous semble suffisamment accessible pour satisfaire les utilisateurs.

Concernant le fonctionnement du logiciel, il sera donc basé sur Unity auquel nous ajouterons des scripts C\#.

Maintenant que nous avons éclairci les objectifs du logiciel, nous nous baserons sur ce rapport comme cahier des charges de référence. Il est évident que n'ayant pas encore pris en main l'ensemble des outils de développement, certains points seront probablement modifiés par la suite.




