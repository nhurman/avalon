\section{Spécifications fonctionnelles du projet}

Dans notre rapport précédent, nous définission un cahier des charges qui s'articulait principalement autour de 3 grands axes qui étaient les suivants : 
\begin{itemize}\renewcommand{\labelitemi}{$\bullet$}
\item Présence d'interactions avec l'environnement
\item Applications aisément portable vers différents systèmes d'utilisation
\item Exploitation de différents périphériques d'entrée/sortie
\end{itemize}
Nous allons donc dans un premier temps présenter les solutions techniques que nous avons retenues pour la réalisation du projet, en détaillant de quelle façon elles répondent à ces 3 problématiques, pour ensuite présenter un diagramme de cas d'utilisation qui résumera le fonctionnement global de l'application. 

\subsection{Solutions techniques retenues}
La réalisation de notre projet s'appuiera donc principalement sur deux environnements de travail, Unity3D et MiddleVR. 

\subsubsection{Unity : le moteur de rendu 3D}
L'utilisation d'Unity était imposée, en vertu des nombreux avantages du logiciel. Unity est un moteur de rendu 3D utilisé principalement pour la réalisation de jeux vidéos, et qui dispose d'une version gratuite. Très complet, il propose les différentes fonctionnalités dont nous avons besoin : il permet d'ouvrir le modèle 3D de l'appartement tremplin qui nous a été fourni par le centre de Kerpape, et de le modifier. 
Il permet aussi de réaliser les interactions entre utilisateur et environnement qui nous intéressent, que ce soit l'appui sur les différents interrupteurs gérant l'éclairage et l'ouverture/fermeture des volets, ou bien l'utilisation de la télécommande dont disposent les résidents de l'appartement qui centralise la plupart des actions qu'ils peuvent effectuer. 
Enfin dernier avantage de Unity, il propose différentes cibles de compilation. En effet, même si dans un premier temps nous allons développer une application qui fonctionnera sous Windows 7 ou 8 avec un ensemble clavier/souris, nous aimerions ne pas être limités à cet environnement par la suite. \newline

Néanmoins, un problème est survenu lors de l'utilisation de Unity : le modèle 3D de l'appartement qui nous avait été fourni avait été créé grâce à 3DSMax, et les deux
\textbf{Unity + Import du modèle / textures (problèmes!)}

\subsubsection{MiddleVR : la gestion des périphériques d'interaction}

MiddleVR répond au troisième des objectifs que nous nous sommes fixés : exploiter différents périphériques d'entrée/sortie. En effet, pour ce faire, il fallait pouvoir s'abstraire desdits périphériques au cours du développement de l'application. L'idée est de développer toutes les fonctionnalités qui nous intéressent, l'utilisation des différents interrupteurs, etc, et d'ensuite disposer d'un moyen facile d'associer l'usage d'un périphérique donné à une fonctionnalité donnée.\newline

Or c'est précisément ce que nous permet MiddleVR. Prenant la forme d'un plugin Unity disponible gratuitement, il est capable de reconnaître les différents périphériques de réalité virtuelle à notre disposition, pour ensuite les associer à certaines actions.
\textbf{Périphériques d’entrée / MiddleVR}

\subsection{Cas d'utilisation}
