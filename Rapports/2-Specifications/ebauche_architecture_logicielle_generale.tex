\section{Architecture générale}
	\subsection{Briques logicielles}
		Notre projet peut être séparé en plusieurs briques logicielles ; celles-ci correspondent à une première ébauche de l'architecture globale du programme.
		\subsubsection{Moteur de rendu : Unity}
			La première brique est Unity; nous l'utiliserons ici pour sa composante moteur de jeu et environnement de développement, et non pour ses fonctionnalités de création d'objets 3d et ... 
		\subsubsection{Gestion des périphériques : MiddleVR}
			(DEJA FAIT en 2)Raison de son utilisation (multiples périphériques …)
		\subsubsection{Interactions avec les objets}
			Nous allons devoir développer un systême d'interractions entres les objets, pour qu'une action sur un objet ait un effet sur un autre objet.
		\subsubsection{Gestion des scénarios}
			Les scénarios demandés dans le cahier des charges devront être implémentés; il pourrait être appréciable que de nouveaux scénarios puissent facilement être ajoutés.
			De ce fait, nous avons besoin d'un système de gestion des scénarios qui devra par exemple donner l'ordre de différentes actions à réaliser par l'utilisateur, et d'éventuelles aides supplémentaires s'il devait réaliser une action autre que celle demandée.
		\subsubsection{Gestions des objets manipulables}
			Nous devrons aussi implémenter de quoi gérer les objets, pour pouvoir les retrouver et les gérer dans l'espace.
		\subsubsection{Menu}
			Un menu devra être disponible au lancement de l'application, ainsi qu'au cours de l'utilisation.
	\subsection{Liens entre les modules}

%MiddleVR <=> Unity
%Gestionnaire d’objets <=> Objets
%Gestionnaire scénarios <=> Gestionnaire d’objets