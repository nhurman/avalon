\section{Architecture g�n�rale}
\subsection{Briques logicielles}
	Notre projet peut-�tre s�par� en plusieurs "briques" logicielles; celles-ci correspondent � une premi�re �bauche de l'architecture globale du logiciel. 
	\subsubsection{Moteur de rendu : Unity}
		La premi�re brique est Unity; nous l'utiliserons ici pour sa composante moteur de jeu et environnement de d�veloppement, et non pour ses fonctionnalit�s de cr�ation d'objets 3d et ... 
	\subsubsection{Gestion des p�riph�riques : MiddleVR}
		(DEJA FAIT en 2)Raison de son utilisation (multiples p�riph�riques �)
	\subsubsection{Interactions avec les objets}
		Nous allons devoir d�velopper un syst�me d'interractions entres les objets, pour qu'une action sur un objet ait un effet sur un autre objet.
	\subsubsection{Gestion des sc�narios}
		Les sc�narios demand�s dans le cahier des charges devront �tre impl�ment�s; il pourrait �tre appr�ciable que de nouveaux sc�narios puissent facilement �tre ajout�s.
		De ce fait, nous avons besoin d'un syst�me de gestion des sc�narios qui devra par exemple donner l'ordre de diff�rentes actions � r�aliser par l'utilisateur, et d'�ventuelles aides suppl�mentaires s'il devait r�aliser une action autre que celle demand�e.
	\subsubsection{Gestions des objets manipulables}
		Nous devrons aussi impl�menter de quoi g�rer les objets, pour pouvoir les retrouver et les g�rer dans l'espace.
	\subsubsection{Menu}
		Un menu devra �tre disponible au lancement de l'application, ainsi qu'au cours de l'utilisation.
\subsection{Liens entre les modules}

MiddleVR <=> Unity
Gestionnaire d�objets <=> Objets
Gestionnaire sc�narios <=> Gestionnaire d�objets